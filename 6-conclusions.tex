% vim: set fenc=utf-8 ft=latex encoding=utf-8
% -*- mode: latex; coding: UTF-8; -*-
%!TEX root = knowledge-curation.tex
\section{Conclusions}
\label{cha:conclusion}

The emerging typology of knowledge types presented in this paper allows to study and characterize Q\&A knowledge dissemination within a community. In this study, we utilized the typology to discover that \SO and \RH mailing list support two distinct approaches for constructing knowledge, participatory knowledge construction and crowd-based knowledge construction, however each channel supports them differently. While participatory knowledge construction is more common in the \RH mailing list, crowd knowledge construction is more prevalent on \SO.

% \dmg{this is the only section I did not touch}

%     Understanding the interplay between channels should be the next step to gain further insights into software development practices.
%     To that end, we contrasted the way knowledge is shared on Stack Overflow and the R-help mailing list, as well as an extensible categorization of Q\&A channel messages that is meant to be used to compare and analyse knowledge in \channels.
%     To conduct a fair comparison, we collected raw data from both channels, transformed this data to a common format through an analysis of the messages to map questions, answers, and other elements.

%     After the analysis of more than 500 threads, we identified five types of messages: question, answer, comment, update, and flag; and more than 35 categories along with their properties, including how-to question, tutorial answer, announcement update, not-an-answer flag, and clarifications comment.

%     We identified two mechanisms for the construction of knowledge: \emph{participatory}, in which answers are created with the collaboration of various users; and \emph{crowd-based}, which is non-collaborative and where solutions are posted without any acknowledgement to previous answers.
%     We also analysed links and how these contribute to the construction of knowledge, such as by referencing source code, projects, and other posts related to a particular question.

%     We also identified user behaviours based on their comments.
%     We noted certain usage characteristics such as solving their own question by providing the answer, and cross-posting in both channels.

%     To bring further insights on our findings, we conducted a survey that collected 26 answers from R users.
%     The result was a list of pros and cons of using both channels. We incorporated them in the set of recommendations for using multiple channels and resources.

%%% Local Variables:
%%% mode: latex
%%% TeX-master: "knowledge-curation.tex"
%%% End:
