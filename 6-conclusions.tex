% vim: set fenc=utf-8 ft=latex encoding=utf-8
% -*- mode: latex; coding: UTF-8; -*-
%!TEX root = knowledge-curation.tex
\section{Conclusions}
\label{cha:conclusion}

The purpose of this study was to understand how the R community collaborates when using different communication channels in the creation and curation of knowledge.
In particular, we concentrated on studying how this community has used Stack Overflow (using the R-tag) and the \RH mailing list to both ask and answer questions, through a random sample of 400 threads from each channel. Our research shows that both channels are active communication channels where participants are willing to help others. 

We found that knowledge contributed in response to a question can be classified into four main categories: answers, updates, flags, and comments. The number of
responses sent in each of these categories was between 1.4 and 2.5 times greater on \SO than on the \RH mailing list. While all four types of contributions exist in both
channels, they exhibit differences. For example, on \SO, answers are more focused towards step-by-step tutorials, while \RH answers are more
likely to be suggestions or alternatives. Similarly, on \SO, updates are focused on language (grammar and spelling), while on \RH, the updates are
expansions on previous responses.

The analysis of these questions and answers shows that knowledge is constructed in each channel in a different manner. On \SO, there is a tendency to use
a crowd approach: participants contribute knowledge independently of each other rather than improve other answers. This is likely a result of the
gamification of \SO where the person who provides the best answer is the one that gains the most points.
In contrast, the \RH mailing list uses a participatory approach where participants are more likely to build on other answers, collaborating towards finding the best solution.

Another important difference between both channels is that \SO focuses on making knowledge available for future retrieval. On the other hand, knowledge on the \RH mailing list 
focuses on the discussion of knowledge, but not in its long-term storage or retrieval. Respondents to our survey commented that while it is easy to find answers on \SO
and make sense of them, on \RH it is not only hard to find the relevant answers to a question, but it is also hard to see how the many responses to a question
relate to each other, and ultimately, what the best answer to the question may be.

Another result of our research is that we found that participants prefer \SO to ask questions that are expected to have a direct answer. They prefer to use the \RH mailing list when the
question requests opinions (\SO forbids them) or when they expect to reach core developers of the R project. Some participants ask the same question in both
channels in the hopes of gaining the advantages of both channels. Additionally, \RH has the ability to complement \SO by providing a medium where the rationale
of answers can be discussed.

Overall, this research shows that the R community is committed to using both channels to help others. Each channel has advantages and disadvantages, and the
community appears to be using both effectively to create and curate knowledge regarding the R language.  We provided \textbf{recommendations} for community members that need to use these or other Q\&A channels.  Furthermore, our \textbf{typology of knowledge artifacts} that we summarized in Table~\ref{table:type-of-knowledge} can be used by other researchers that wish to study and understand how knowledge is constructed and curated in other channels or across other communities.  As new channels (such as Slack) become more widely adopted, studying these newer channels and comparing them to existing channels is an imperative aspect of understanding knowledge formation in software development. %something we need to continue studying.


%By helping newcomers, the R-community has played an important role in the diffusion and popularity of the language.

%and it participates actively in both Stack Overflow and its own mai


% The emerging typology of knowledge types presented in this paper allows to study and characterize Q\&A knowledge dissemination within a community. In this study, we utilized the typology to discover that \SO and \RH mailing list support two distinct approaches for constructing knowledge, participatory knowledge construction and crowd-based knowledge construction, however each channel supports them differently. While participatory knowledge construction is more common in the \RH mailing list, crowd knowledge construction is more prevalent on \SO.

%\alexey{Need help here. Not sure what else to say. perhaps something about how this can help others or future work?}

% \dmg{this is the only section I did not touch}

%     Understanding the interplay between channels should be the next step to gain further insights into software development practices.
%     To that end, we contrasted the way knowledge is shared on Stack Overflow and the R-help mailing list, as well as an extensible categorization of Q\&A channel messages that is meant to be used to compare and analyse knowledge in \channels.
%     To conduct a fair comparison, we collected raw data from both channels, transformed this data to a common format through an analysis of the messages to map questions, answers, and other elements.

%     After the analysis of more than 500 threads, we identified five types of messages: question, answer, comment, update, and flag; and more than 35 categories along with their properties, including how-to question, tutorial answer, announcement update, not-an-answer flag, and clarifications comment.

%     We identified two mechanisms for the construction of knowledge: \emph{participatory}, in which answers are created with the collaboration of various users; and \emph{crowd-based}, which is non-collaborative and where solutions are posted without any acknowledgement to previous answers.
%     We also analysed links and how these contribute to the construction of knowledge, such as by referencing source code, projects, and other posts related to a particular question.

%     We also identified user behaviours based on their comments.
%     We noted certain usage characteristics such as solving their own question by providing the answer, and cross-posting in both channels.

%     To bring further insights on our findings, we conducted a survey that collected 26 answers from R users.
%     The result was a list of pros and cons of using both channels. We incorporated them in the set of recommendations for using multiple channels and resources.

%%% Local Variables:
%%% mode: latex
%%% TeX-master: "knowledge-curation.tex"
%%% End:
