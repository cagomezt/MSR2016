% vim: set fenc=utf-8 ft=latex encoding=utf-8
% -*- mode: latex; coding: UTF-8; -*-
%!TEX root = knowledge-curation.tex
\section{Background}
\label{cha:background}
We begin with an overview of the R community and describe in more detail the two main channels used for asking and answering questions by R community members.

% The following parts below are not really helpful. Some parts might go into the intro (like Vasilecu), but others should be just removed. I'm commenting this out - Alexey.


% The emergence of new \textit{\channels} (e.g., wikis, forums, and Q\&A websites) and \textit{communities of practice} have affected the way
%     software developers communicate.
%     Project-related activities are scattered among many channels like bug trackers, wikis, and source code repositories and
%     using multiple \channels have become a standard among programmers. 
%     Awareness of emergent \channels and learning how to use them are challenges that developers have to overcome~\cite{Storey2014}.

%     Research has studied \channels use extensively \cite{Pal2011a,Pal2012a, Jin2013, Jiang2013,German2013,Sowe2008a, Singh2009, Parnin2013}. \dmg{any more
%       citations?} However, most studies concentrate on how a specific team uses \channels. Today, developers are likely to participate (actively or passively)
%     in large communities that relate to their professional needs. These communities, each usually concentrating in a particular domain (such as a tool,
%     programming language, technology, etc), contribute to the knowledge at the disposal of its members.  As Vasilescu stressed~\cite{Vasilescu2014b}, there is a need to analyse and compare how a community
%     of software developers uses \channels to create, curate knowledge relevant to its members\dmg{am i correct in the interpretation of the citation?}
% \gpoo{Vasilescu did not go that far: ``...to better understand the effects associated with a transition from mailing lists to social QA and, e.g., whether mailing lists will eventually die off, future research could also consider analysing the content of the discussions from the two venues...''}.
%     Understanding how such communities use specific \channels is
%     key to improve developer practices regarding communication, coordination, and knowledge sharing, specially beyond their primary collaborators.

%     Only few researchers have investigated these topics.  Bird et all studied how activity in mailing lists is correlated to activities in the source
%     code~\cite{Bird2006}; Storey et all have studied the role of social media in software engineering~\cite{Storey2014, Storey2010}; Kavaler contrasted APIs use
%     and questions asked on Stack Overflow~\cite{Kavaler2013} while Vasilescu et al studied the interplay between Stack Overflow and the software development
%     process, reflected on changes committed in a source code management system~\cite{Vasilescu2013a}.

% \subsection{\Channel}

%     For the context of this paper we define a \channel as {``a method or system by which information is communicated or distributed to others using different means''}.
%     A \channel is composed of users, messages, and a channel.
%     \textit{Users} are responsible for the creation of messages.
%     \textit{Messages} contain the knowledge transmitted to the receiver and take different forms depending on the channel's characteristics (text, graphics, video, sound, or a combination of them).
%     A \textit{channel} provides a method or system to coordinate, communicate, collaborate and share knowledge with other users~\cite{Storey2014}.
%     Depending on the characteristics of a channel, some tasks are easier to accomplish than others~\cite{Storey2014,Vasilescu2014b}.

% \subsection{Community of practice}

%     % What it is
%     A community of practice is \textit{``a group of people who share a concern or a passion for something they do and learn how to do it better as they interact regularly''}~\cite{Wenger2000}.
%     In contrast with formal work groups and project teams, community members are part of the community by their own will~\cite{Wenger2000}.
%     Members work towards a common objective, learning and helping each other in the process.

%     % Components
%     The core components of a community of practice are the domain, the practice, and the community~\cite{Wenger2011}.
%     The \textit{domain}, or shared interest, defines the identity of the community.
%     The \textit{practice} identifies members of a community as \textit{practitioners} that are constantly developing and sharing a set of resources (tools, documentation, histories, or experiences) to address recurring problems. 
%     The \textit{community}, comprises the activities in which members discuss to help each other, enabling them to learn from the community.

%     % Why are they important
%     Community of practices are important because they help members to solve problems quickly, transfer best practices, develop professional skills, identify experts, form social bounds between members, and drive strategies~\cite{Wenger2011, Storey2014}.
%     Given the proper structure, practitioners can be the best option to manage the construction of knowledge~\cite{Wenger2011}.

% \subsection{Section about knowledge in software?}
% Perhaps discuss what is knowledge in software development, what do we mean when we talk about knowledge and knowledge types --- we talk about externalized knowledge, and more specifically, knowledge in the form of question and answers. 

% \subsection{Section about other studies the community/knowledge through channels}
% Perhaps talk about other studies where researchers studied a community of developers through channels. Vasilescu has a paper about GitHub and StackOverflow, and his thesis I think is about StackOverflow and Mailing list (if not, then one of his other papers).

% Has anyone else mapped messages to knowledge types?

\subsection{The R Community of Practice}
    
    The R project\footnote{\url{https://www.r-project.org/}} was born in 1993 as a free and open source programming language and software environment for statistical computing, bioinformatics, and graphics~\cite{Ihaka1996}.
    The R community is composed of two groups:
    \begin{enumerate*}[label=(\arabic*)]
      \item \textit{R-core}, a team of 20 software developers that maintain and evolve the R language, and
      \item \textit{Periphery}, which includes everyone else (language users and package developers).
    \end{enumerate*}

%The size of the R community has a network effect. As the number of its members grows, the awareness of the language increases, jobs are created, available resources increase their size, depth and quality, new tools and libraries appear. In essence, helping individuals cope with R benefits the entire R community: its members learn and/or improve their expertise, the experts gain reputation for their knowledge and willingness to help, the R-users base grows and with it, its cloud and market.

    The R community is an eclectic open source community that goes beyond software development 
    and includes biologists and statisticians with no or limited programming experience.
    Its entire history of mailing list communication is archived and publicly available.
    The R community has also been the subject of extensive research in community evolution~\cite{German2013,Vasi1escu2014PhD} and the interplay between channels~\cite{Vasilescu2014c}.

    Our study focused on the analysis of \SO and the R-help mailing list, two of the main channels in the R community.
    We chose them because they are the main channels that provide Q\&A support to the community.

\subsubsection{R-help Mailing List}
    There are several mailing lists to help R community members solve programming problems with the R language: \emph{R-help}, \emph{R-package-devel}, \emph{R-devel}, \emph{R-packages}, \emph{R-announce} and \emph{Bioconductor}. However, the R-help mailing list is the main channel for discussing problems and solutions using R.  Other messages are also encouraged, such as documentation, benchmarks, examples, and announcements.
%    It is oriented to users interested in R, but not necessarily with programming skills.

    The R-help mailing list used to be the main \channel for asking and answering questions within the R community, but a significant number of users migrated to Stack Overflow~\cite{Vasilescu2014c}.
    Despite the reduced number of users, the R-help mailing list is still very active---on average, a subscriber may receive 55 emails a day.

\subsubsection{Stack Overflow}
\label{subsec:Rtag}

    In contrast to the R-help mailing list, Stack Overflow incorporates a rich visual and user-friendly interface with social media and gamification features.
    The social aspect of the Website improves participation and provides strong support for creating and sharing knowledge as well as encouraging informal mentorship~\cite{Jenkins2009, Storey2014}.
    Meanwhile, its gamification features provide a system based on reputation points and badges to reward user participation and earn them points that enable functionality inside the site.
    It has been reported that \SO's gamification mechanisms boost participation~\cite{Vasi1escu2014PhD} and enable mutual assessment~\cite{Singer2013}.


\subsection{\SO vs. Mailing Lists}
Software development is a knowledge-building process~\cite{naur1985programming}. Due to the emergence of socially-enabled tools and channels and the formation of communities of practice~\cite{Storey2014}, it is important to understand how knowledge is created and shared within these communities. In our study, we focus on knowledge in the form of questions and answers within the R community.

%\cite{Tausczik2014} studied math overflow and found that some activities are individual and some interindividual. but only studied math overflow.

As part of a study on the transition to gamified environments, Vasilescu~\cite{Vasi1escu2014PhD} examined the popularity of Stack Exchange (including the \SO R tag) and
mailing lists within the R community. He found that since 2010, the number of message threads has decreased on the \RH mailing list, while the number of
R-related questions asked on the Stack Exchange network has increased. Our study also examined \SO's R tag and the \RH mailing list, but we aimed to understand the \textit{knowledge types} used. This allows us to characterize the different knowledge seeking and sharing approaches on each channel. Vasilescu also examined the \textit{difference in activity} between contributions made by members active on both channels and members focused on a single medium. We also found members of the R community that were active on both channels, however, we aimed to understand \textit{why members post to a particular channel}.

Similar to Vasilescu, Squire~\cite{Squire2015a} studied a project's \textit{transition to the \SO gamified channel}. She focused on examining whether four software projects that moved from mailing lists to \SO showed improvements in terms of developer participation and response time. She found that all four projects showed improvements on \SO compared to mailing lists, however, she also found that several projects have moved back to using mailing lists despite achieving these improvements. In our study, we found that both channels have knowledge support for question and answers, however, there are important differences between the two channels. For example, \SO's competitive environment creates an incentive to be the first to answer rather than improve other answers and participate in discussions.

%%% Local Variables:
%%% mode: latex
%%% TeX-master: "knowledge-curation.tex"
%%% End:
