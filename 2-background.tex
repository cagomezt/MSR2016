% vim: set fenc=utf-8 ft=latex encoding=utf-8
% -*- mode: latex; coding: UTF-8; -*-
%!TEX root = knowledge-curation.tex
\section{Background}
\label{cha:background}

\dmg{Section 2 (before 2.1) can be trimmed a lot. It is very fuzzy in content.}

    The emergence of new \textit{media channels} (e.g., wikis, forums, and Q\&A websites) and \textit{communities of practice} have caused a stir in the industry.
    Project-related activities are scattered among many channels like bug trackers, wikis, and source code repositories.
    Many projects are now global and open to the public through online repositories, collaboration is not limited by geographical barriers, and a new type of programmer has emerged: \textit{the social programmer}.

    Awareness is one of the main issues that social programmers face on a daily basis.
    The variety of channels available imposes the social programmer to use multiple channels in unison~\cite{Storey2010, Storey2014}.
    There is a need to analyse and compare media channels and the way programmers use them~\cite{Vasilescu2014b}.
    Understanding channels is key to improve the developer practices, communication, coordination, and knowledge sharing.

    The research community have identified various aspects of media channels within communities of practice.
    We have algorithms to detect experts on social channels~\cite{Pal2011a,Pal2012a}, models that explain the propagation of information through channels~\cite{Jin2013, Jiang2013}, an understanding of the relationships between the evolution of the community and its products~\cite{German2013}, and discovered ways that social programmers are using media channels~\cite{Sowe2008a, Singh2009, Parnin2013}.

    Some issues are still pending.
    Issues that current programmers need to understand, including the synergy between media channels and the way these channels affect communities of practices.
    To our knowledge, just a few researchers have investigated these topics.
    We know the activities in mailing lists are correlated to activities in the source code~\cite{Bird2006}, the role of social media in software engineering~\cite{Storey2014, Storey2010}, a complementary perspective on using APIs and the questions asked on Stack Overflow~\cite{Kavaler2013}, and the interplay between Stack Overflow and the software development process, reflected on changes committed in a source code management system~\cite{Vasilescu2013a}.

\subsection{Media Channel}

    A media channel is \textit{``a method or system by which information is communicated or distributed to others using different means''}\footnote{Based on the Oxford dictionary definitions for \emph{media} and \emph{channel}.}.
    A media channel is composed of users, messages, and a channel.
    \textit{Users} are the active part and are responsible for the creation of messages.
    \textit{Messages} contain the knowledge transmitted to the receiver and take different forms depending on the channel's characteristics (text, graphics, video, sound, or a combination of them).
    A \textit{channel} provides a method or system to coordinate, communicate, collaborate and share knowledge with other users~\cite{Storey2014}.
    Depending on the characteristics of a channel, some tasks are easier to accomplish than others~\cite{Storey2014,Vasilescu2014b}.

\subsection{Community of practice}

    % What it is
    A community of practice is \textit{``a group of people who share a concern or a passion for something they do and learn how to do it better as they interact regularly''}~\cite{Wenger2000}.
    In contrast with formal work groups and project teams, community members are part of the community by their own will~\cite{Wenger2000}.
    Members work towards a common objective, learning and helping each other in the process.

    % Components
    The core components of a community of practice are the domain, the practice, and the community~\cite{Wenger2011}.
    The \textit{domain}, or shared interest, defines the identity of the community.
    The \textit{practice} identifies members of a community as \textit{practitioners} that are constantly developing and sharing a set of resources (tools, documentation, histories, or experiences) to address recurring problems. 
    The \textit{community}, comprises the activities in which members discuss to help each other, enabling them to learn from the community.

    % Why are they important
    Community of practices are important because they help members to solve problems quickly, transfer best practices, develop professional skills, identify experts, form social bounds between members, and drive strategies~\cite{Wenger2011, Storey2014}.
    Given the proper structure, practitioners can be the best option to manage the construction of knowledge~\cite{Wenger2011}.

\subsection{The R community of practice}
    
    The R project\footnote{\url{https://www.r-project.org/}} was born in 1993, as a Free and Open Source programming language and software environment for statistical computing, bioinformatics, and graphics~\cite{Ihaka1996}.
    The R community is composed of two groups:
    \begin{enumerate*}[label=(\arabic*)]
      \item \textit{R-core}, a team of 20 software developers that maintains and evolves the R language, and
      \item \textit{Periphery} includes everybody else (language users and package developers).
    \end{enumerate*}

    The R community is an eclectic open source community beyond software development.
    % It provides broader relevance outside the software development community, since it
    It includes biologists and statisticians, with no or limited programming experience.
    Its entire history of mailing list communication is archived and publicly available.
    The R community has also been the subject of extensive research in community evolution~\cite{German2013} and the interplay between channels~\cite{Vasilescu2014c}.

    In our study, we focused the analysis on the R-help mailing list and Stack Overflow, both channels are among the main ones in the R community.
    We chose them because their description are similar in terms of the community support.

\subsubsection{R-help Mailing List}
    The R community has a group of mailing lists for helping community members to solve programming problems with R language: \emph{R-help}, \emph{R-package-devel}, \emph{R-devel}, \emph{R-packages}, and \emph{Bioconductor}.

    In particular, the R-help mailing list is to discuss problems and solutions using R. 
    Other messages are also encouraged, such as documentation, benchmarks, examples, and announcements not covered by \emph{R-announce} or \emph{R-packages}.
    It is oriented to users interested in R, but not necessarily with programming skills.

    The R-help mailing list used to be the main media channel for asking and answering questions within the R community, but a significant number of users migrated to Stack Overflow~\cite{Vasilescu2014c}.
    Despite the reduced number of users, the R-help mailing list is still very active; on average, a subscriber may receive 55 emails a day.

\subsubsection{Stack Overflow}
\label{subsec:Rtag}

    In contrast to the R-help mailing list, Stack Overflow incorporates a rich visual and user-friendly interface with social media and gamification features.
    The social aspect of the website improves participation and provides strong support for creating and sharing knowledge as well as encouraging informal mentorship~\cite{Jenkins2009, Storey2014}.
    Meanwhile, gamification provides a system based on reputation points and badges to reward users' participation, thus earning points that enable functionality inside the site.
    Gamification mechanisms boost participation~\cite{Vasilescu2014} and enable mutual assessment~\cite{Singer2013}.

    The adoption of social media has occurred at a much faster rate than any previous communication technology \cite{Chui2012}.
    In the last decade, Stack Overflow has become the most popular media channel for answering software development related questions, nearly replacing previous methods of communication that accomplished the same objective~\cite{Vasilescu2014c}.
    Despite Stack Overflow's advantages over Q\&A mailing lists such as the R-help (i.e., gamification environment and rich visual user interface), there are still many users who prefer the latter.
    We will learn the way programmers use Stack Overflow and the R-help mailing list to gain and share knowledge.

%%% Local Variables:
%%% mode: latex
%%% TeX-master: "knowledge-curation.tex"
%%% End:
