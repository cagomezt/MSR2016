\documentclass{sig-alternate-05-2015}
\usepackage[T1]{fontenc}
\usepackage[utf8]{inputenc}
\synctex=-1 
\usepackage[english]{babel}

\usepackage{rotating}

%\usepackage[numbers]{natbib} % back referencing 
\usepackage[hidelinks]{hyperref} % back-referencing + hyperlinks
%\usepackage{etoolbox} % back referencing
\usepackage{paralist}
\usepackage{listings} % soruce code
\usepackage{array}
\usepackage{soul}
\usepackage{tablefootnote}
\usepackage{adjustbox} % resize boxes (for resizing arrays)
\usepackage{bookmark} % add bookmark to a PDF
\usepackage{pdfpages}
\usepackage{pifont}
%\usepackage{amsthm}
\usepackage{pdflscape} % landscape images

\usepackage{xspace}
\usepackage{subscript}
\usepackage{multirow}
\usepackage{subfig}
\usepackage{graphicx}
\usepackage{comment}
\usepackage{tabularx}

%\usepackage[normalem]{ulem}
%\usepackage{fixltx2e}
%\usepackage{tabularx}
\usepackage{booktabs}
\usepackage{pslatex}
%\usepackage{mathptmx,helvet,courier}

% Long Table and decimal aligned columns
%\usepackage{dcolumn}
%\usepackage{longtable}


\newcommand{\mytitle}{Knowledge Curation in a Developer Community: \\A Study of Stack Overflow and Mailing Lists}
\newcommand{\myauthor}{Carlos Arturo G{\'o}mez Teshima}
\newcommand{\mysubject}{}
\newcommand{\mykeywords}{Knowledge, Case study, media channels}

% Comments
% commands for inserting comments
\usepackage{color}
\usepackage[normalem]{ulem} % for \sout (strikethrough o tachado)
\usepackage{ifthen}
\usepackage{comment}
\newboolean{showcomments}
\setboolean{showcomments}{true} % toggle to show or hide comments
\ifthenelse{\boolean{showcomments}}
  {
		%\usepackage{showkeys} %Show lables and refs
		\newcommand{\nbb}[2]{
		% \fbox{\bfseries\sffamily\scriptsize#1}
		\fcolorbox{black}{yellow}{\bfseries\sffamily\scriptsize#1}
		{\sf$\blacktriangleright$\textcolor{blue}{\textit{#2}}$\blacktriangleleft$}
		% \marginpar{\fbox{\bfseries\sffamily#1}}
		}

        \newcommand{\todo}[1]{\textcolor{red}{{\sc #1}}}
        \newcommand{\internalnote}[1]{\marginpar{\scriptsize note: #1}}
        \newcommand{\version}{\emph{\scriptsize{$-$\today$-$}}}
		\newcommand{\remarks}[1]{\color{red}[#1]\color{black}}
		\newcommand{\modified}[1]{\color{blue}[#1]\color{black}}
		\newcommand{\raw}{$\rightarrow$}
		\newcommand{\ins}[1]{\textcolor{blue}{\uline{#1}}} % please insert
		\newcommand{\del}[1]{\textcolor{red}{\sout{#1}}} % please delete
		\newcommand{\chg}[2]{\textcolor{red}{\sout{#1}}{\raw}\textcolor{blue}{\uline{#2}}} % please change
		\newcommand{\ugh}[1]{\textcolor{red}{\uwave{#1}}} % please rephrase
  }
  {
		\newcommand{\nbb}[2]{}
        \newcommand{\todo}[1]{}
        \newcommand{\internalnote}[1]{}
		\newcommand{\remarks}[1]{}
		\newcommand{\modified}[1]{#1}
		\newcommand{\version}{}
		\newcommand{\ugh}[1]{#1} % please rephrase
		\newcommand{\ins}[1]{#1} % please insert
		\newcommand{\del}[1]{} % please delete
		\newcommand{\chg}[2]{#2} % please change
  }


\begin{document}

% Copyright
\setcopyright{acmcopyright}
%\setcopyright{acmlicensed}
%\setcopyright{rightsretained}
%\setcopyright{usgov}
%\setcopyright{usgovmixed}
%\setcopyright{cagov}
%\setcopyright{cagovmixed}


% DOI
%\doi{10.475/123_4}

% ISBN
%\isbn{123-4567-24-567/08/06}

%Conference
\conferenceinfo{MSR '16}{May 14--15, 2016, Austin, TX, USA}

%\acmPrice{\$15.00}

%
% --- Author Metadata here ---
\conferenceinfo{MSR}{'16 Austin, Texas USA}
%\CopyrightYear{2007} % Allows default copyright year (20XX) to be over-ridden - IF NEED BE.
%\crdata{0-12345-67-8/90/01}  % Allows default copyright data (0-89791-88-6/97/05) to be over-ridden - IF NEED BE.
% --- End of Author Metadata ---

%
% General information
%


\title{\mytitle}

\numberofauthors{6}

\author{
    \alignauthor
    Carlos Gómez Teshima\\
    \affaddr{University of Victoria}\\
    \affaddr{Victoria, BC, Canada}\\
    \email{teshima@uvic.ca}
    \alignauthor 
    Alexey Zagalsky\\
    \affaddr{University of Victoria}\\
    \affaddr{Victoria, BC, Canada}\\
    \email{alexeyza@uvic.ca}
    \and
    \alignauthor
    Margaret-Anne Storey\\
    \affaddr{University of Victoria}\\
    \affaddr{Victoria, BC, Canada}\\
    \email{mstorey@uvic.ca}
    \alignauthor 
    Daniel M. German\\
    \affaddr{University of Victoria}\\
    \affaddr{Victoria, BC, Canada}\\
    \email{dmg@uvic.ca}
    \alignauthor
    Germán Poo-Caamaño\\
    \affaddr{University of Victoria}\\
    \affaddr{Victoria, BC, Canada}\\
    \email{gpoo@uvic.ca}
}

\date{\the\year}

\maketitle

\begin{abstract}
One of the many effects of social media in software development is the flourishing of very large communities of practice where members share a common interest, such as programming languages, frameworks, and tools. These communities of practice use many different \channels and little is known about how these communities create, share, and curate knowledge using such channels.

In this paper, we report a qualitative study of how one community of practice---the R software development community---creates and curates knowledge associated with questions and answers (Q\&A) in two of its main \channels: the R-tag in Stack Overflow and the R-users mailing list. The results reveal that knowledge is created and curated in two main forms: participatory, where multiple members explicitly collaborate, and crowdsourced, where individuals mostly work independently of each other. The contribution of this paper is a characterization of the types of knowledge that are exchanged by these communities of practice, including a description of the reasons why members choose one channel over the other.  Finally, this paper enumerates a set of recommendations to assist practitioners in the use of multiple channels for Q\&A.
\end{abstract}

%
% The code below was generated with the tool at
% http://dl.acm.org/ccs.cfm
%
\begin{CCSXML}
<ccs2012>
<concept>
<concept_id>10003120.10003130.10003131.10003235</concept_id>
<concept_desc>Human-centered computing~Collaborative content creation</concept_desc>
<concept_significance>500</concept_significance>
</concept>
</ccs2012>
\end{CCSXML}

\ccsdesc[500]{Human-centered computing~Collaborative content creation}

%
% End generated code
%

%
%  Use this command to print the description
%
\printccsdesc

\keywords{\mykeywords}

% vim: set fenc=utf-8 ft=latex encoding=utf-8
% -*- mode: latex; coding: UTF-8; -*-
%!TEX root = knowledge-curation.tex
\section{Introduction}
\label{cha:introduction}


    % Role of media channels
    Media channels play an important role in today's knowledge economy, as well as the collaboration, coordination, and communication activities that occur between programmers.
    Media channels are more than just delivery systems---they connect users with a \textit{community of practice} or groups of people with a common interest.
    Two popular media channels used by software developer are Stack Overflow\footnote{\url{http://stackoverflow.com/}} and mailing lists.
    Selecting the most appropriate media channel to transmit an idea can be challenging, given the variety of equally suitable tools and sites.
    To decide, a programmer considers the characteristics will benefit most among channels.
    Such considerations are: experts on the channel, flexibility on topics allowed, if the channel is asynchronous, socially enabled, or has gamification elements~\cite{Vasilescu2014c}.

    We investigated the way \textit{knowledge} (or user generated content) is curated within a particular software development community.
    For this study we chose the R community, since it provided broader relevance outside the software development community by including users with no or limited programming experience (e.g., biologist or statisticians).
    Our overarching goal was to provide tools for further studies that analyse and compare the knowledge flowing through media channels.
    Thus, the research question investigated are:

\dmg{what about Rephrasing RQ to stress differences between channels also?}
    \begin{enumerate}[label=\bfseries{RQ-\arabic*.},itemsep=3pt, topsep=2pt, leftmargin=3em, parsep=0pt]
        \item What types of knowledge are shared on Stack Overflow and the R-help mailing list within the R community?
        \item How is the knowledge constructed on Stack Overflow and the R-help mailing list? 
        \item Why do certain users post to both Stack Overflow and the R-help mailing list?
    \end{enumerate}

    We analysed the main Q\&A channels related to programming that the R community contains: R-help mailing list and Stack Overlow.
    To analyse the channels, we applied a qualitative \textit{exploratory case study} methodology.
    We also conducted a survey to bring further insights on the findings.
    We constructed a series of categories that supports knowledge classification and knowledge comparison of the main type of messages which these two channels provided.
    Based on the knowledge categories analysis, we compared the way knowledge was shared on Stack Overflow and the R-help mailing list.
    Finally, we extracted a set of recommendations to assist in the usage of multiple Q\&A channels, and when linking resources that are external to both channels.

\dmg{Add a summary of the results}

%%% Local Variables:
%%% mode: latex
%%% TeX-master: "knowledge-curation.tex"
%%% End:

% vim: set fenc=utf-8 ft=latex encoding=utf-8
% -*- mode: latex; coding: UTF-8; -*-
%!TEX root = knowledge-curation.tex
\section{Background}
\label{cha:background}
    % What I pretend with this part without title is to provide a non detailed introduction to all the rest of the chapter that put in context the reader. 
    % about the evolution of the channels and the ine
%   Prior to the 21st century, books and classrooms were the main way to learn new programming languages and to answer questions.
%   Software development was an activity performed by small geographically co-located groups using email and phone calls as the main way to coordinate activities, ask questions, collaborate with others, and share knowledge~\cite{Storey2014}.

\dmg{Section 2 (before 2.1) can be trimmed a lot. It is very fuzzy in content.}

    The emergence of new \textit{media channels} (e.g., wikis, forums, and Q\&A websites) and \textit{communities of practice} have caused a stir in the industry.
    Project-related activities are scattered among many channels like bug trackers, wikis, and source code repositories.%~\cite{Guzzi2013}.
%    Learning new programming languages is a just-in-time activity performed with the help of online resources~\cite{Sim2013,Storey2010,Hartmann2008}.
    Many projects are now global and open to the public through online repositories, collaboration is not limited by geographical barriers, and a new type of programmer has emerged: \textit{the social programmer}.

    Awareness is one of the main issues that social programmers face on a daily basis.
    The variety of channels available imposes the social programmer to use multiple channels in unison~\cite{Storey2010, Storey2014}.
There is a need to analyse and compare media channels and the way programmers use them~\cite{Vasilescu2014b}.
Understanding channels is key to improve the developer practices, communication, coordination, and knowledge sharing.
    
%    Regardless of how social programmers select their preferred channels, they have to invest time in learning the way each channel works.
%    Also, channels are becoming increasingly complex with more options for communicating, making media literacy a complex issue.
    %information is available everywhere but the quality of it is also a concern that the social programmer have to deal with.

%    Multiple studies on software development media channels report the way channels are used~\cite{Guzzi2013, Storey2014, Singer2014},  topic trends~\cite{Barua2012, Kavaler2013, Wang2013d}, best practices~\cite{Asaduzzaman2013, Treude2011, Allamanis2013}, and \textit{social programmer} behaviours~\cite{Lang2013}.
%    Studies on the interplay between channels report insights on channel migration processes~\cite{Vasilescu2014c}, synergy between channels~\cite{Vasilescu2013a, Bird2006, Kavaler2013}, and channel usage~\cite{Stolee2010,Storey2014}.
%   Questions raised that still beg unanswered are: Is one communication channel replacing the other, or are they cooperating?, Why communities have more than one channel to solve the same problem? In which circumstances one communication channel should be used over another?.

    The research community have identified various aspects of media channels within communities of practice.
    We have algorithms to detect experts on social channels~\cite{Pal2011a,Pal2012a}, models that explain the propagation of information through channels~\cite{Jin2013, Jiang2013}, an understanding of the relationships between the evolution of the community and its products~\cite{German2013}, and discovered ways that social programmers are using media channels~\cite{Sowe2008a, Singh2009, Parnin2013}.

    Some issues are still pending.
    Issues that current programmers need to understand, including the synergy between media channels and the way these channels affect communities of practices.
    To our knowledge, just a few researchers have investigated these topics.
    We know the activities in mailing lists are correlated to activities in the source code~\cite{Bird2006}, the role of social media in software engineering~\cite{Storey2014, Storey2010}, a complementary perspective on using APIs and the questions asked on Stack Overflow~\cite{Kavaler2013}, and the interplay between Stack Overflow and the software development process, reflected on changes committed in a source code management system~\cite{Vasilescu2013a}.

\subsection{Media channels}

    % What is a media channel?
    The Oxford dictionary\footnote{\url{http://oxforddictionaries.com/definition/english}} defines medium as \textit{``a mean by which something is communicated or expressed''}, and a channel as \textit{``a method or system for communication or distribution''}.
    Combined, a media channel is \textit{``a method or system by which information is communicated or distributed to others using different means''}.

    A media channel is composed of users, messages, and a channel.
    \textit{Users} are the active part and are responsible for the creation of messages.
    \textit{Messages} contain the knowledge transmitted to the receiver and take different forms depending on the channel's characteristics (text, graphics, video, sound, or a combination of them).
    A \textit{channel} provides a method or system to coordinate, communicate, collaborate and share knowledge with other users~\cite{Storey2014}.

Depending on the characteristics of a channel, some tasks are easier to accomplish than others.
For instance, Stack Overflow is changing the way in which developers learn, communicate, collaborate, and share knowledge among themselves~\cite{Storey2014}.
Stack Overflow may even replace the mailing list usage as a consequence of its gamification system, rich interface, and social media features~\cite{Vasilescu2014b}.

%   Some research have focused on different components and aspects of channels, such as,
%categorize questions according to their topic~\cite{Treude2011}, determine the characteristics of unanswered questions~\cite{Asaduzzaman2013}, study the way messages are disseminated on social coding sites~\cite{Jiang2013}, and propose a method to quantify the risk of not having maintainers for code implemented in a certain programming language~\cite{Vasilescu2013b}.

\subsection{Community of practice}

\dmg{Tie 2.2 and 2.3 better. You don't use community of practice again until page 4!}
    % What it is
    A community of practice is \textit{``a group of people who share a concern or a passion for something they do and learn how to do it better as they interact regularly''}~\cite{Wenger2000}.
    In contrast with formal work groups and project teams, community members are part of the community by their own will~\cite{Wenger2000}.
    Members work towards a common objective, learning and helping each other in the process.

    % Components
    The core components of a community of practice are the domain, the practice, and the community~\cite{Wenger2011}.
    The \textit{domain}, or shared interest, defines the identity of the community.
    The \textit{practice} identifies members of a community as \textit{practitioners} that are constantly developing and sharing a set of resources (tools, documentation, histories, or experiences) to address recurring problems. 
    The \textit{community}, comprises the activities in which members discuss to help each other, enabling them to learn from the community.

    % Why are they important
    A community of practice is more than the sum of its parts.
    It helps members to solve problems quickly, transfer best practices, develop professional skills, identify experts, form social bounds between members, and drive strategies~\cite{Wenger2011, Storey2014}.
    It also accumulates and updates knowledge through practitioners~\cite{Wenger2010}, enabling them to take a collective responsibility for managing the knowledge according to their needs~\cite{Wenger2011}.
    Given the proper structure, practitioners can be the best option to manage the construction of knowledge~\cite{Wenger2011}.

    % How they change
%   Communities of practice are like living organisms, evolving and adapting according to their context, producing new tools for the community, and external sites.
%   Communities change their practices and structure regularly while adapting dynamically to new situations.
%   For example, Mozilla adopted the Mercurial tool~\cite{Rodriguez-Bustos2012} and changed their version release strategy~\cite{Khomh2012} as a way to keep up with a fast changing business environment.

\subsection{The R community}

The R project\footnote{\url{https://www.r-project.org/}} was born in 1993, as a Free and Open Source programming language and software environment for statistical computing, bioinformatics, and graphics~\cite{Ihaka1996}.
%R is an implementation of the S programming language combined with lexical scoping inspired by Scheme.
%   It was created by Ross Ihaka and Robert Gentleman and is now developed by the R Development Core Team.
The R community is composed of two groups:
\begin{enumerate*}[label=(\arabic*)]
  \item \textit{R-core}, a team of 20 software developers that maintains and evolves the R language, and
  \item \textit{Periphery} includes everybody else (language users and package developers).
\end{enumerate*}

    The R community is an eclectic open source community beyond software development.
    % It provides broader relevance outside the software development community, since it
    It includes biologists and statisticians, with no or limited programming experience.
    Its entire history of mailing list communication is archived and publicly available.
    The R community has also been the subject of extensive research in community evolution~\cite{German2013} and the interplay between channels~\cite{Vasilescu2014c}.

    In our study, we focused the analysis on the R-help mailing list and Stack Overflow, both channels are among the main ones in the R community.
%   As media channels, the R-help mailing list and Stack Overflow provide similar benefits to the R community (i.e., Stack Overflow\footnote{\url{http://stackoverflow.com/tour}} and R-help\footnote{\url{https://www.r-project.org/mail.html}}).
%   The R-help mailing list and Stack Overflow are one of the many channels available within the R community.
    We chose them because their description are similar in terms of the community support.

\subsubsection{R-help Mailing List}
    The R community has a group of mailing lists for helping community members to solve programming problems with R language: \emph{R-help}, \emph{R-package-devel}, \emph{R-devel}, \emph{R-packages}, and \emph{Bioconductor}.
%   Through email, R users can send their questions to different mailing lists depending on the topic.
%   Members subscribed to the R mailing lists can contribute by answering the user directly or posting to the list.
%   In the last case, the email subject is kept as an identifier for the reader.

In particular, the R-help mailing list is to discuss problems and solutions using R. 
Other messages are also encouraged, such as documentation, benchmarks, examples, and announcements not covered by \emph{R-announce} or \emph{R-packages}.
It is oriented to users interested in R, but not necessarily with programming skills.
%   As a mailing list, R-help does not provide a user interface to manage the email threads.

    The R-help mailing list used to be the main media channel for asking and answering questions within the R community, but a significant number of users migrated to Stack Overflow~\cite{Vasilescu2014c}.
    Despite the reduced number of users, the R-help mailing list is still very active; on average, a subscriber may receive 55 emails a day.

\subsubsection{Stack Overflow}
\label{subsec:Rtag}

    In contrast to the R-help mailing list, Stack Overflow incorporates a rich visual and user-friendly interface with social media and gamification features.
    The social aspect of the website improves participation and provides strong support for creating and sharing knowledge as well as encouraging informal mentorship~\cite{Jenkins2009, Storey2014}.
    Meanwhile, gamification provides a system based on reputation points and badges to reward users' participation, thus earning points that enable functionality inside the site.
%   For example, 20 points allow users to participate in the site's chat rooms, 100 points allow users to edit wiki posts, 2000 points allow users to edit questions and answers, and with 25000 points, users can access site analytics.
%   Stack Overflow also provides trophies for display in users' profiles\footnote{\url{http://stackoverflow.com/help/badges}}, and a bounty reputation system to attract the interest of unanswered questions.
    Gamification mechanisms boost participation~\cite{Vasilescu2014} and enable mutual assessment~\cite{Singer2013}.

%   \begin{figure}[!htb]
%   \centering
%   \includegraphics[width=\columnwidth]{Figures/SOInterface_A}
%   \caption{Stack Overflow interface [Question section]}
%   \label{fig:SOInterface_A}
%   \end{figure}

%   Stack Overflow's interface is rich with information. Figures \ref{fig:SOInterface_A} and \ref{fig:SOInterface_B} depict the interface separated into two sections. Figure \ref{fig:SOInterface_A} describes the post in relation to the \textit{question}.
%   The elements are numbered from 1 to 8, and are described as follows:
%   (1) title of the question;
%   (2) number of votes for the question, as well as two arrow buttons to vote up (positive) or down (negative);
%   (3) a star button to mark the question as a favourite and the number of users who have done it;
%   (4) tags applied to the question;
%   (5) a button to add a short, text-based comment to the question;
%   (6) body of the question which might contain, along with the description, other aids such as images, source code, examples, and links;
%   (7) last user that edited the question along with their reputation points;.
%   and (8) information of the user who posted the question, including alias, silver and copper badges, and the date of the posted question.

%   \begin{figure}[!htb]
%   \centering
%   \includegraphics[width=\columnwidth]{Figures/SOInterface_B}
%   \caption{Stack Overflow interface [Answer Section]}
%   \label{fig:SOInterface_B}
%   \end{figure}

%   Figure \ref{fig:SOInterface_B} shows the post in relation to the \textit{answer} located below the question in the interface.
%   The elements are numbered 1 to 8, and are described as follows:
%   (1) number of answers provided to the question;
%   (2) sorting buttons to display the answers by latest activity, oldest first, or most recent first;
%   (3) number of votes for the answer, as well as two buttons (up and down arrows) to allow users to vote up (positive) or down (negative);
%   (4) a check mark to indicate that the owner of the question marked the answer as the solution to the question;
%   (5) body of the answer which might contain, along with the proposed solution, other aids such as images, source code, examples, and links;
%   (6) last user that edited the question along with their reputation points;
%   (7) information about the user who posted the question, including alias, silver and copper badges, and the date of the posted question;
%   and (8) comments to the answer, which are fairly short and limited to include only text.

    The adoption of social media has occurred at a much faster rate than any previous communication technology \cite{Chui2012}.
    In the last decade, Stack Overflow has become the most popular media channel for answering software development related questions, nearly replacing previous methods of communication that accomplished the same objective~\cite{Vasilescu2014c}.
    %<<
%   Figure~\ref{fig:VasilescuFA1-1} shows the number of questions asked each month on Stack Overflow, Cross Validate and the R-help mailing list, and Figure~\ref{fig:VasilescuFA1-2} shows the number of questions answered on the R-help mailing list (after September 2008) and Stack Exchange each month.
    %>>
    Despite Stack Overflow's advantages over Q\&A mailing lists such as the R-help (i.e., gamification environment and rich visual user interface), there are still many users who prefer the latter.
    We will learn the way programmers use Stack Overflow and the R-help mailing list to gain and share knowledge.

%   \begin{figure}
%       \centering
%       \begin{subfigure}[b]{\columnwidth}
%         \includegraphics[width=\columnwidth]{Figures/VasilescuFA1}
%          \caption{Questions asked (threads started) per month on R-help and Stack Exchange (Stack Overflow and Cross Validated).}
%         \label{fig:VasilescuFA1-1}
%        \end{subfigure}
%       \begin{subfigure}[b]{\columnwidth}
%         \includegraphics[width=\columnwidth]{Figures/VasilescuFA2}
%          \caption{Questions answered on R-help mailing list (after September 2008) and Stack Exchange per month: participants exclusive to the mailing list versus those also active on Stack Exchange.}
%         \label{fig:VasilescuFA1-2}
%        \end{subfigure}
%       \caption{Number of questions asked and answered~\cite{Vasilescu2014c}.}
%       \label{fig:VasilescuFA1}
%   \end{figure}

%%% Local Variables:
%%% mode: latex
%%% TeX-master: "knowledge-curation.tex"
%%% End:

%!TEX root = knowledge-curation.tex
\subsection{that other section - will be merged with previous subsection}

% Most of the content below has been included in the intro (as motivation), some parts should be added to the subsection about R in the Background

The R language continues to grow in popularity, and with it, the size of its community of practice. The recent TIOBE index for Programming Languages ranks it
number 19 among all languages.  According to a recent survey, R has become the highest paying skill in IT. The interest in R continues to grow.
Early in 2016, Microsoft announced support for R in Visual Studio~\cite{RMicrosoft2016}.

Being an open source project, without commercial backing, the R community has played an important role in its diffusion. Those new to the R language have
numerous resources to learn the language and receive help: mailing lists, blogs, books, online- and offine--courses, questions-and-answers sites (such as
\SO). In all, these resources provide a vast and rich corpus of knowledge. The R-community benefit from this corpus, but it also the one that drives its
creation and curation.  For example, in July of 2009, during the Open Source Convention (OSCON'09), the organizers of a birds-of-a-feather session invited users
to participate in a Flashmob to seed Stackoverflow with R related questions~\cite{OSCONRFlashMob2009}. The premise of the session was that \SO lacked R-related
content. The organizers have gathered a list of commonly asked questions from the R-help mailing list, Rseek.org and a survey of R-users. Its impact was
noticed, and \SO acknowledge to ``officially condone'' the practice~\cite{SOFlashMob2009}.

The size of the R community has a network effect. As the number of its members grows, the awareness of the language increases, jobs are created, available
resources increase their size, depth and quality, new tools and libraries appear. In essence, helping individuals cope with R benefits the entire R
community: its members learn and/or improve their expertise, the experts gain reputation for their knowledge and willingness to help, the R-users base grows and
with it, its cloud and market.

Without a single entity directing and controlling the R-language, knowlege in R has grown organically from its community. Knowledge is exchanged and curated in
many \channels (emails, blogs, books, presentations, web sites, etc). Like any other community of practice, the R-community takes advantage of available \channels
to achieve this goal.
Two \channels are at the center of this process: R-help mailing list and \SO. The R-users mailing list was established \dmg{when} as way to assist those using
the language. \SO is not specifically oriented towards R, but its section dedicated to the language has grown rapidly\footnote{\href{http://r-bloggers.com/r-is-the-fastest-growing-language-on-stackoverflow/}{http://r-bloggers.com/r-is-the-fastest-growing-language-on-stackoverflow/}}.

Without a doubt, \SO has changed, for the better, the way programmers seek knowledge. \SO can play the role of a expert-on-call, who is capable---and willing--
to answer questions of any difficulty level about any programming technology (R included). The gamification features of \SO have also guaranteed the willingness of experts to answer
those questions---frequently within minutes of being posted. Equally important is the ability of \SO users to curate the knowledge being created, making sure that
the best answers surface to the top, and become a valuable asset to those seeking the same answer in the future. \SO has become an effective tool to create, curate and exchange knowledge, including knowlege or R.

One would expect that the traffic in the R-users \ml would have begin to fizzle as \SO popularity increses. If \SO is so effective at matching those who seek
knowlege with those that have it, doesn't that obviate the need for the R-users mailing list? at least regarding questions and answers. Yet, that does not appear
to be the case. The R-help \ml continues to grow in traffic, implying that there it is still an important resource for the R-community. It appears as if R-users
and \SO complement each other.

There exist obvious inherent differences between both \channels. Mailing lists unite users by subscription, creating a tight community, and their content lacks
organization (except for its natural organization provided by the metadata of the emails---subject, threading, authors, date) and are not optimized for long
term storage and retrieval.  \SO, on the other hand, is a more loose community and it is opmitized for the curation and long term storage of the knowledge.

However, little is known of the actual differences of use between both \channels. 
In particular, how the types questions-and-answers seeked in one channel compare to
the other, why users choose one channel over the other, why some users participate in both channels and what are the perceptions that its participants have
regarding each \channel.

The objective of this study is to emprically compare how knowledge, specifically knowledge manifested as questions-and-answers, is seeked, shared and curated in
both, the R-users mailing list and the R section of \SO.

\dmg{needs more}



%%% Local Variables:
%%% mode: latex
%%% TeX-master: "knowledge-curation.tex"
%%% End:

% vim: set fenc=utf-8 ft=latex encoding=utf-8
% -*- mode: latex; coding: UTF-8; -*-
%!TEX root = knowledge-curation.tex
\section{Methodology}
\label{cha:methodology}

    We carried out a qualitative case study of the knowledge that flows through the channel. 
    This method is considered to be suitable for studies that explore under-researched phenomena and for providing an in-depth analysis within its real-life context~\cite{Yin2009}.

%	This chapter describes the elements of the methodology, including the research questions, the adopted case study methodology, and the phases of the study.
%	This chapter also outlines the procedure used to collect and analyse the data in this study.

%\subsection{Research Questions}

%	The four research questions that guided this thesis are:

%\paragraph*{RQ-1. What types of knowledge are shared on Stack Overflow and the R-help mailing list within the R community?}

%	In the R community, the R-help mailing list serves the same purpose as Stack Overflow.
%	This led to the question of \emph{what types of knowledge are shared on Stack Overflow and the R-help mailing list?}
%	To answer this question I proceeded to analyse and categorize the knowledge in questions, answers, updates, comments and flags on Stack Overflow and the R-help mailing list.
%	Based on the analysis I was able to contrast the way knowledge flows through Stack Overflow and the R-help mailing list.

%\paragraph*{RQ-2. How is the knowledge constructed on Stack Overflow and the R-help mailing list?}

%	As discussed before, Stack Overflow and the R-help mailing list support the R community. 
%	Such a statement implies that the interactions hosted by these two media channels are of a collaborative nature.
%	I wondered if the same applies to the creation and sharing of knowledge in these two channels.
%	My goal was to identify the mechanisms and strategies on Stack Overflow and the R-help mailing list used to construct knowledge collaboratively and individually (if any).

%\paragraph*{RQ-3. How does the sharing of links on Stack Overflow and the R-help mailing list support knowledge construction?}

%	On the Internet, links support the reuse and referencing of data from other resources.
%	Links contain information that is valuable for messages, and depending on how they are used, links can support knowledge sharing practices in different ways.
%	For instance, a link can expand what is known about a topic by referencing more complete sources of information, or provide data to reproduce certain behaviours on source code examples.
%    Previously, I have identified the types of links on Stack Overflow and how they support diffusion of knowledge~\cite{Gomez2013}.
%	For this study, I pursued the identification of how links contribute to the construction of knowledge.
%	Thus, I categorized links posted in the body of messages on Stack Overflow and the R-help mailing list based on their type (e.g., Q\&A Website, and Forums), and how each type of link supported the knowledge construction.

%\paragraph*{RQ-4. Why do certain users post to both Stack Overflow and the R-help mailing list?}

%	As mentioned by Vasilescu~\cite{Vasilescu2014b}, there is a group of users that are active on Stack Overflow and the R-help mailing list. 
%	I wondered if there were any advantages or disadvantages on using both channels.
%	With that in mind, I identified a list of active users in both channels and used open coding methods to analyse their posts.

%\subsection{Case Study Methodology}
%
%	A case study facilitates the exploration of a phenomenon within its context using a variety of data sources~\cite{Yin2009,Yin2012}.
%	In software engineering, a \textit{case study} is defined as  \textit{``an empirical enquiry that draws on multiple sources of evidence to investigate one instance (or a small number of instances) of a contemporary software engineering phenomenon within its real-life context, especially when the boundary between phenomenon and context cannot be clearly specified''}~\cite{Runeson2012}.
%
%	Yin~\cite{Yin2012} states that a case study should be used when: 
%	(1) ``How'' or ``why'' questions are trying to be answered; 
%	(2) the researcher cannot manipulate the behaviours of those involved in the study; 
%	(3) the context is an important part of the study; 
%	(4) there is no clear difference in what is happening between the phenomenon and the context; 
%	and (5) when multiple sources of evidence have to be covered.
%	These conditions apply to the nature of this study and its research questions, and motivated the selection of the case study methodology for this work.
%	Specifically, this work is an exploratory case study to explain the interplay of multiple media channels within a community in terms of the knowledge created and shared.
%	%Our goal is to gain insights and compare the usage of selected media channels within the R community. \remarks{Check if this is the same everywhere}
%
%	The study is divided in two phases that were performed in parallel: mining of data archives, and the survey.
%	Figure \ref{fig:StudyPhases} depicts the general organization of the study design.
%	In the next sections of this chapter, each phase is explained in detail.

%	\begin{figure} [!ht]
%		\centering
%		\includegraphics[width=\columnwidth]{Figures/StudyPhases}
%		\caption{General overview of the study design}
%		\label{fig:StudyPhases}
%	\end{figure}

\subsection{Phase 1: Mining data archives} 
\label{sec:studyDesign}

	The mining of the data archives method involved a three step process: data collection, data analysis, and reporting.%~\cite{Runeson2012}.
	The \textbf{data collection} step consisted in gathering the body of data required for analysis.
	This data was a selected set of R-related posts from Stack Overflow and the R-help mailing list.
	In the \textbf{data analysis} step, we analysed the data by looking for answers for the research questions.
	Finally, the \textbf{report} step, we consolidated the results, which are presented in section~\ref{cha:findings}.

\subsubsection{Data collection and preparation}
\label{subsec:preparation}

	Stack Overflow and the R-help mailing list store their messages in publicly available archives.
	The records available for Stack Overflow start in 2008 (the birth of Stack Overflow), while the R-help archives go back to 1997.
	To make both data sets comparable, we analysed the data from 2008 until 2013, a period of time that both channels were available simoultaneously.
For Stack Overflow, we obtained a data dump file available in its website.
For the R-help mailing list data, we retrieved the archives available as MBOX files from the R website.

	We used two different software tools to prepare the data.
	\begin{enumerate*}[label=(\arabic*)]
	\item to process the Stack Overflow data, we used a modified version of Sam Saffron's application, So-Slow\footnote{\url{https://github.com/SamSaffron/So-Slow}}; and,
	\item to process the R-help mailing list archives, we wrote a software application\footnote{Our tool is available at \url{https://github.com/cagomezt/GTMail}}, based on the Bettenburg \textit{et al}~\cite{Bettenburg2009} recommendations of how to process mailing list data.
	\end{enumerate*}
    Once we pre-processed the data, we stored them in a database for further analysis.

%	The following subsections detail the analysis process for each media channel.

%\subsubsection{The R-help mailing list}
%\label{subsec:r-help}

To ensure accurate results when processing the R-help mailing list, we followed the series of recommendations proposed by Bettenburg \textit{et al}.~\cite{Bettenburg2009}.
To make the data comparable againt the Stack Overflow dat set, we transformed the email addresses to MD5 hashes, and changed the time zone of the mailing list messages (UTC+2) to the time zone used by Stack Overflow (UTC).

%\subsubsection*{Stack Overflow}
Stack Exchange releases a new data dump from all their websites every three months\footnote{\url{http://stackexchange.com/sites}}.
However, the last dump file that containing email addresses as MD5 hashes was released in September 2013.
Since then, Stack Overflow does not provide the email addresses.
Because of this, we used the data dump file from September 2014, but updated the table \texttt{users} with the hashes in the dump file from September 2013, for whose \texttt{ID}s were identical in both data sets.
If an user from the 2013 data file did not exist in the 2014 data, we ignored it.
%	As stated previously, I used a modified version of Sam Saffron's application, So-Slow.
%	The purpose of this is to extract the information in the file using XML tags (e.g., post, user, and comment), and load it in a database.
We filtered all R-related data by selecting only messages with the R tag (\texttt{r}) and its synonyms\footnote{\url{http://stackoverflow.com/tags/r/synonyms}} (\texttt{rstats} and \texttt{r-language}).

	Table~\ref{table:data} depicts a summary of the data uploaded in the database.
	The R-help has more questions, answers, and users than Stack Overflow, because there were approximately ten years of additional data.
	Only Stack Overflow's data contains ``comments'' information. %so this field is empty for the R-help mailing list column.

	\begin{table}[!htb]
	  \centering
      \caption{Raw data collected for each channel.}
      \begin{small}
        \begin{tabular}{lrr}
	        \toprule
	        Type          &  R-help & Stack Overflow \\
	        \midrule
	        Questions     & 101,931 &  67,393 \\
	        Answers       & 213,366 &  99,620 \\
	        Comments      &       - & 286,124 \\
	        Users         &  39,150 &  26,324 \\
	        \bottomrule
        \end{tabular}
      \end{small}
	  \label{table:data}
	\end{table}


%\subsubsection*{Data merging}
%	There are some studies that propose different techniques for merging users' identities by analysing the data from multiple repositories (e.g., mailing lists, bug tracking information, and source code management tools)~\cite{Bird2006, Kouters2012,Vasilescu2014c}.
%	Bird \textit{et al}.~\cite{Bird2006} proposed a heuristic to match users' identities across multiple mailing list archives by combining parts of user names and email addresses. For example, the \textit{cagomezt} prefix is likely to belong to \textit{Carlos Arturo Gomez Teshima}.
%	Furthermore, Kouter \textit{et al}.~\cite{Kouters2012} used a natural language processing technique called Latent Semantic Analysis to merge identities on very noisy data.
%	However, it has been demonstrated that all existing approaches produce false positives and false negatives~\cite{Goeminne2013}.

To merge both datasets, we matched Stack Overflow's email MD5 hashes with the MD5 hash version of email addresses from the R-help mailing list data.
We did not use any method to infer email addresses based on user names.
%In practice, this conservative approach~\cite{Vasilescu2014b}
The resulting set was 1,421 different users with the same email address on both media channels.

%	Because Stack Overflow only provides the email addresses as MD5 hashes, and to make both data sets comparable, the mailing list emails were converted to their corresponding MD5 hashes.

Although MD5 hashes are not \textit{collision resistant} and could possibly lead to false positives, it is unlikely that two different email addresses share a MD5 hash.
    The probability to find a MD5 collision is less than $1/2^{64}$~\cite{Rivest1992}.
	%\footnote{\url{https://en.wikipedia.org/wiki/Collision_resistance}}, and therefore, this could possibly lead to false positive resistant outcomes.

\subsubsection{Data analysis process}
\label{sec:dap}

To analysis the data that flows through Stack Overflow and the R-help mailing list, we performed a qualitative and exploratory approach, as it best suits research when a concept or phenomenon requires more understanding, with little pre-existing research~\cite{Creswell2009}.

In particular, we performed an inductive approach~\cite{Runeson2012} to analyse the data from Stack Overflow and the R-help mailing list.
This is an iterative process, where across the study is necessary to switch between data selection and data analysis, or between data reporting and data collection.
As advised to reduce bias~\cite{Runeson2012}, two researchers conducted the analysis, both computer scientists with a background in qualitative data analysis.

\begin{figure*}[!htb]
	\centering
	\includegraphics[width=.9\textwidth]{Figures/CodingExample}
	\caption{Example of data coding. Each row is a thread message. Questions, comments, and answers are identified with the number on the first column. Columns in yellow contain the codification for each message type. The last two columns contain the memos and the URL.}
	\label{fig:CodingExample}
\end{figure*}

%	%<<
%	\begin{figure} [!htb]
%		\centering
%		\includegraphics[width=\columnwidth]{Figures/ContentAnalysisFlow_3}
%		\caption[Our content analysis method]{Qualitative approach used to analyse Stack Overflow and the R-help mailing list data. The chart shows the process and techniques (coloured figures) used to analyse and develop the findings.}
%		\label{fig:ContentAnalysisFlow}
%	\end{figure}
%	%>>

%Figure \ref{fig:ContentAnalysisFlow} depicts a visual explanation of the data analysis process for this study.
%The coloured shapes depict the techniques used to support the data analysis, which are explained as follow:

The techniques used to support the data analysis, which are explained as follow:

	\begin{description}[itemsep=3pt, topsep=2pt, leftmargin=3em, parsep=0pt]
		\item[Memoing] is the act of taking notes (coding) on what the researcher is learning from the data during the analysis~\cite{Groenewald2008}.% for example, the hypotheses regarding a code, and relationships between concepts.
We wrote reflective memos in a spreadsheet next to the applicable codes (see figure \ref{fig:CodingExample}).
These memos were used to create the codes, and hypotheses about the relationships between concepts.

		\item[Affinity diagrams] allows to organize ideas and data into groups, and to find the relationships between concepts~\cite{Scupin1997}.
		We used them to discuss new insights, and while defining categories and relationships between them.

		\item[Inter-rater agreement \textit{Cohen Kappa}] is a coefficient used to measure the agreement between two coders who classify items into mutually exclusive categories~\cite{Stemler2004}.
		Ladis and Koch suggest that values above 0.60 or 60\% to obtain substantial results~\cite{Landis1977}.
		In a previous study~\cite{Gomez2013}, we used the same coefficient to measure agreement between coders.
		Based on this experience, we set a value above 0.80 or 80\% as the minimum to obtain substantial results.
		We used this coefficient after each coding session as a way to trigger discussion.

		\item[Code book] is a book that contains the definitions of the codes that researchers look during the data analysis~\cite{MacQueen1998}.
%		Codes are the building blocks for theory and foundations on which the researcher's argument rest.
%		We coded an initial set of 120 threads over three sessions.
%		In each session, we separately coded 40 threads.
		We coded in multiple sessions, which allowed us to refine the definitions.
		Each entry is associated with a title, a formal definition, an example, and space for notes from the researcher.
		The final version of the code book is detailed in section~\ref{cha:findings}.
	\end{description}

%\paragraph{The Analysis Process}

	The focus of the analysis is to \textit{understand the context of the media channels and the community}.
	The process consisted of:
	First, a recollection of the official information for both channels and the community to build a background of the community of practice and the channels studied.
%    From the channels, I collected posting guides, rules, channel objectives, and competitors, whereas from the community I collected the number of members, how it works, and the media channels that the community uses.
	Second, a mapping between messages from Stack Overflow with messages on the R-help mailing list.
    This is to overcome how the data is structured in both channels.
%    Stack Overflow has a clear delimitation of what is a question, an answer, a comment, a flag and an update, while the R-help mailing list is just plain text.
    The mapping of messages between both channels was as follows:

	\begin{description}[itemsep=3pt, topsep=2pt, leftmargin=3em, parsep=0pt]
		\item[Question:] the message is the first on the thread, and it contains the main question.
		\item[Answer:] the message provides a solution to the main question of the thread.
	 	\item[Update:] the message claims for a modification to a question (or answer) made by the author of such a question (or answer).
		\item[Comment:] the message offers a clarification to a specific part of the question or answer.
		\item[Flag:] the message requests attention from the moderator (e.g., repeated questions, spam, or rude behaviour).
	\end{description}

	To answer RQ1 and RQ2, we queried the database by selecting a random number of threads of each channels within a time frame.
	%<<
	The data set was capped at 400 threads for each channel when we deemed our observations as being saturated.
	To answer RQ3, we added a condition that matched, on both channels, messages with the same subject written by the same author.
	%>>
	The results were 79 threads, and therefore, we analysed the entire population available. 

%	To code the data, we used an \textit{open coding} technique that involves reporting \textit{what the researcher saw} during each coding session. 
%	The researcher has to keep in mind, all the time, the objective of the study and perform the coding based on it.
%	Each researcher coded the data on a separated way.
%	During the coding session, we wrote memos as needed, and marked repetitive patterns.
%	Later, we met to compare and discuss findings, and begin developing codes.

%	From our initial codes, we began the process of creating a \textit{coding book} to outline definitions. 
%	This set of codes were used later during the \textit{selective coding} step.
%	At this point, the researcher stops coding every occurrence, and begins seeing larger trends and connections within the data and codes.
%	It is possible that during the \textit{selective coding} step some codes have to be reformulated, or maybe split into more codes. 
%	Also, it is possible to formulate completely new codes as needed.
%	Whenever there is a new code or any is changed, it is necessary to go back and recode the material.

%	As a coding tool, my colleague and I used a spreadsheet in which each row represents a message of the thread.
%	%<<
%	If for any reason a message appeared to fit in more than one category, each researcher selected, at their own discretion, a primary category to represent the message.
%	%>>
%	Figure~\ref{fig:CodingExample} depicts an example of the coding spreadsheet that we used.
%	The number in the first column identifies if the message is a question, an answer or a comment. 
%	For instance, if the number assigned to the question was ``1'', then the answers were enumerated with consecutive numbers separated by a point (e.g., 1.1); and the comments were enumerated in a similar way to enumerated answers, but using three numbers: the first number represents the question, the second represents the answer, and the third represents the comment consecutive (e.g., 1.0.1). 
%	The second column contains the message, the third column the channel, the fourth column the question categorization, and so on.
%	The last column contains the URL to the thread on the channel.
%	Inside each cell, a semicolon ($;$) represents a sub-category, and the double pipe ($||$) divides two different ideas (e.g., in the \textit{MEMOS} column), or indicates that a message was re-classified after an update (e.g, \textit{ANSWER} column).

%	At the beginning of the coding, before we created the code book, the spreadsheet had only the \textit{ID}, the \textit{MESSAGE}, the \textit{MEMOS}, and the \textit{URL} columns.
%	During each iteration, the spreadsheet was updated with the classification and type of messages that my colleague and I were defining.

%	Originally, both researchers read the threads directly from the spreadsheet.
%	However, this method of reading turned out uncomfortable, and we fell back to read the threads directly from each channel rather than the spreadsheet.


\subsection{Phase 2: The survey} 

In \textit{phase 2}, we conducted a survey\footnote{A copy of the survey is available at \url{http://goo.gl/mxmH5J}} with members of the R community with the purpose of obtaining additional insights on the findings.
To test and refine the questions, format and tone, we performed two pilots.
%	First, I created a draft of the survey and did two pilots:
%	(1) with colleagues in our research group, and
%	(2) with R users at the University of Victoria.
%	The objective was to test and refine the questions, tone, rankings, and the format of the survey.
%	The survey questions were structured into five sections:
%	(1) the user,
%	(2) Stack Overflow use,
%	(3) the R-help mailing list use,
%	(4) Stack Overflow and the R-help mailing list if both used, and
%	(5) resources linked to used.
%	Survey's sections 2, 3 and 4 would only become active if the participant was a user of the channel.

We announced our survey on Twitter, Reddit, the R-help mailing list, and Meta Stack Exchange to reach users of both channels, and minimize the selection bias.
However, on Stack Exchange the announcement was not well received and therefore was deleted a few minutes later after posting it.
We received 26 valid responses out of 32 from the R community members.
%The survey did not collect any personal information.

%%% Local Variables:
%%% mode: latex
%%% TeX-master: "knowledge-curation.tex"
%%% End:

% vim: set fenc=utf-8 ft=latex encoding=utf-8
% -*- mode: latex; coding: UTF-8; -*-
%!TEX root = knowledge-curation.tex
\section{Findings}
\label{cha:findings}
This study focused on how knowledge---in the form of questions and answers---is created, shared, and curated. We first identified and categorized the main types of knowledge contained in the \RH mailing list and in \SO messages with the R tag (RQ1). The emerging categories formed a typology, and allowed us to identify and describe two approaches for constructing knowledge that are supported by these channels (RQ2). Interestingly, we found that some developers are active on both channels, and in some cases, even post the same questions. As a result, we investigated the benefits they gain by doing so (RQ3). In this section, we present our findings.
%This section presents the findings of our study that answer each research question.

\subsection{What Types of Knowledge Are Shared on \SO and the \RH Mailing List}
%\subsection{RQ-1. \rqa}
\label{cha:findings-types}

To answer RQ1, we randomly sampled 400 threads of messages from both \SO and the \RH mailing list, where each thread included a question and the associated responses. We identified five main types of knowledge:
\begin{enumerate*}[label=(\arabic*)]
\item Questions
\item Answers
\item Updates
\item Flags
\item Comments
\end{enumerate*}
Each of these types is further divided into sub-types: Table~\ref{table:type-of-knowledge} presents the typology knowledge types, their description, and the frequency in the sample of 400 threads. As mentioned in Section~\ref{cha:methodology}, this sample provides a reliability of 95\% $\pm$ 5\%. Using the Chi-square test of independence, we tested whether the distribution of types and sub-types of questions were different between the two channels.  In
all cases, they were found to be statistically different (with $\rho \ll 0.001$ in all cases).
    \begin{table*}[!htb]
      \centering
      \caption{Typology of knowledge types found on both \SO (SO) and \RH (RH), and their frequency in the analyzed sample. Numbers in bold represent the most significant differences between the two sets.}
      \begin{small}
\begin{tabular}[h]{p{2.3cm}p{10.3cm}rrrrr}
 && \textbf{SO}                                                                                                                                              & \textbf{RH}  & \textbf{Prop SO} & \textbf{Prop RH}                \\
\toprule
\multicolumn{2}{@{}l}{\textbf{Questions}}\\
  \emph{How-to}                   & Asks how to do something specific.                                                                                                                        & {166}          & {103}              & \textbf{41.50\% }       & \textbf{25.75\%}        \\
  \emph{Bug/Error\-/Exception}    & Asks for a solution or reasons for an error message.                                                                                                       & 27           & 48               & 6.75\%         & 12.00\%        \\
  \emph{Discrepancy}              & Asks about an unexpected result of a specific function, process, or package.                                                                              & 53           & 88               & \textbf{13.25\%}        & \textbf{22.00\%}        \\
  \emph{Set-up}                   & Asks for possible ways to set up the R environment before or after deployment.                                                                            & 15           & 31               & 3.75\%         & 7.75\%         \\
  \emph{Decision help}            & Asks for advice in making a decision.                                                                                                                     & 36           & 35               & 9.00\%         & 8.75\%         \\
  \emph{Conceptual\-/Guidance}    & Asks for conceptual clarification or guidance on topics related to R or statistics.                                                         & 48           & 49               & 12.00\%        & 12.25\%        \\
  \emph{Code reviewing}           & Asks for a code review, explicitly or implicitly.                                                                                                           & 34           & 21               & 8.50\%         & 5.25\%         \\
  \emph{Non-functional}           & Asks for help (or suggestions) with a non-functional requirement such as performance or memory usage.                                                   & 14           & 11               & 3.50\%         & 2.75\%         \\
  \emph{Future reference}         & Asks a question (often self-answering it) that might not exist on the channel, but that is interesting enough to warrant a thread for future reference.         & 5            & 4                & 1.25\%         & 1.00\%         \\
  \emph{Other}                    & Asks for assistance unrelated to the channel, or the message contains unrelated information (e.g., announcements, ideas for improvement).                  & 2            & 10               & 0.50\%         & 2.50\%         \\\cline{3-6}
                                  &                                                                                                                                                          & {400} & {400}     & {100\%} & {100\%} \\
\hline
  \multicolumn{2}{@{}l}{\textbf{Answers}}                                                                                                                                                                                                                          \\
  \emph{Redirecting}                & Provides a link to an existing solution that is not in the thread (e.g. external application, tutorial, project).                                     & 163          & 87               & 20.20\%        & 15.03\%        \\
  \emph{Tutorial}                   & Provides a set of steps to teach people how to solve the issue.                                                                                          & 105          & 15               & \textbf{13.01\%}        & \textbf{2.59\% }        \\
  \emph{Source code}                & Provides a source code snippet as the solution without an extensive explanation about the answer.                                                                   & 198          & 102              & 24.54\%        & 17.62\%        \\
  \emph{Clue/Suggestion/Hint}       & Provides a possible way(s) to fix the issue without actually solving it.                                                                                                     & 43           & 105              & \textbf{5.33\% }        & \textbf{18.13\% }       \\
  \emph{Alternative}                & Provides a different approach to a solution that is related to but not exactly what is being asked (e.g. mathematical approach, data structure modification). & 33           & 98               & \textbf{4.09\% }        & \textbf{16.93\% }       \\
  \emph{Explanation}                & Provides an explanation of an approach that answers the question and lists steps on how to do it.                                                                          & 203          & 101              & 25.15\%        & 17.44\%        \\
  \emph{Announcement}               & Provides a notification about some artifact (e.g., packages, libraries).                                                                                 & 8            & 33               & 0.99\%         & 5.70\%         \\
  \emph{Benchmark}                  & Provides a benchmark of multiple solutions posted by others or compares different answers.                                                               & 5            & 3                & 0.62\%         & 0.52\%         \\
  \emph{Opinion}                    & Provides an opinion or an expansion of another answer by including scenarios and examples.                                                                    & 49           & 35               & 6.07\%         & 6.04\%         \\\cline{3-6}
                                    &                                                                                                                                                          & \textbf{807} & \textbf{579}     & {100\%} & {100\%} \\
\hline
  \multicolumn{2}{@{}l}{\textbf{Updates}}                                                                                                                                                                                                                          \\
  \emph{Announcement}               & Announces specific events (e.g., bounties, future updates).                                                                                              & 27           & 3                & 4.40\%         & 1.21\%         \\
  \emph{Background}                 & Adds additional context to the question or answer .                                                                                                       & 74           & 57               & 12.07\%        & 23.08\%        \\
  \emph{Correction}                 & Corrects format, grammar, spelling, and semantic mistakes.                                                                                               & 301          & 2                & \textbf{49.10\% }       & \textbf{0.81\% }        \\
  \emph{Expansion}                  & Expands the question or answer by providing scenarios or examples.                                                                                       & 116          & 83               & \textbf{18.92\% }       & \textbf{33.60\%}        \\
  \emph{Explanation}                & Explains or clarifies a specific point in the question or answer, such as why the user chose a specific data structure, or the meaning of a variable.    & 83           & 95               & 13.54\%        & 38.46\%        \\
  \emph{Solution}                   & The user answers their own question.                                                                                                                     & 12           & 7                & 1.96\%         & 2.83\%         \\\cline{3-6}
                                    &                                                                                                                                                          & \textbf{613} & \textbf{247}     & {100\%} & {100\%} \\
\hline
  \multicolumn{2}{@{}l}{\textbf{Flags}}                                                                                                                                                                                                                            \\
  \emph{Off-topic/Opinion}          & Identifies questions that are unrelated to the channels' interests or which answers seek opinion.                                                      & 22           & 19               & 27.16\%        & 35.19\%        \\
  \emph{Not an answer}              & Emphasizes \textit{alternative answers} out of the scope of the question, or identifies that a solution does not answer the question.                    & 0            & 27               & \textbf{0.00\% }        & \textbf{50.00\%}        \\
  \emph{Repeated question}          & Notifies a user that the question has been answered previously.                                                                                                & 48           & 8                & \textbf{59.26\%}        & \textbf{14.81\%}        \\
  \emph{Too localized}              & Questions that are too specific and might not help future readers.                                                                                    & 6            & 0                & 7.41\%         & 0.00\%         \\
  \emph{Unclear}                    & Questions that are difficult to understand.                                                                                                              & 5            & 0                & 6.17\%         & 0.00\%         \\\cline{3-6}
                                    &                                                                                                                                                          & {81}  & {54}      & {100\%} & {100\%} \\
\hline
  \multicolumn{2}{@{}l}{\textbf{Comments}}                                                                                                                                                                                                                     \\
  \emph{Clarification}          & Provides (or requests) additional information about a question or answer.                                                                                & 98           & 28               & 17.44\%        & 10.49\%        \\
  \emph{Expansion}              & Provides additional information.                                                                                                                         & 127          & 65               & 22.60\%        & 24.34\%        \\
  \emph{Correction/Alternative} & Suggests a change to a question or answer, offers an alternative solution or a correction.                                                               & 102          & 89               & \textbf{18.15\%}        & \textbf{33.33\% }       \\
  \emph{Compliment/Critic}   & Posts something good, offers thanks, provides an opinion or criticism.                                                                           & 157          & 52               & 27.94\%        & 19.48\%        \\
  \emph{External reference}     & References an external resource.                                                                                                                         & 78           & 33               & 13.88\%        & 12.36\%        \\\cline{3-6}
                                &                                                                                                                                                          &{562}  & {267}     & {100\%} & {100\%} \\
  \bottomrule
        \end{tabular}
      \end{small}
      \label{table:type-of-knowledge}
\vspace{-3mm}
    \end{table*}

\paragraph*{Questions and Answers}
    Questions express one or more problems or concerns faced by a user on the \RH mailing list or \SO, whereas answers represent solutions to questions.  We observed that the types of questions on \SO are more specific than those on the \RH mailing list and are more likely to be tutorials. Also, \SO has more answers per question---2 per question compared to 1.4 for \RH. However, \RH questions tend to offer more suggestions or alternatives than \SO answers. This might be a result of the narrowness of \SO questions.

\paragraph*{Updates}
An update is a modification of a question or answer. On \SO, updates are presented in one of two ways:
\begin{description}[itemsep=3pt, topsep=2pt, leftmargin=1em, parsep=0pt]
\item[Labeled updates] are explicitly shown in the body of questions or answers next to a label that identifies the update (e.g., edit, update, and p.s.).\cassie{what's ps?}
  When multiple update labels appear in a message, each label is accompanied by a number (e.g., \textit{``[Edit 1:]''}), a date (e.g., \textit{``Edit/Update (April 2011):''}), or a bulleted list
  (e.g., ``EDIT: - anova... -drop1...'').

\item[Non-labeled updates] are only visually recognizable through the message history system. The only indication of the change is a box at the end of the
  message that contains the user who performed the change and the date when it happened.
\end{description}
We found that \textit{Non-labeled} updates are often used to correct formatting, grammar, semantic mistakes, and spelling, or to incorporate explanations, examples, and suggestions without changing the meaning of the question or answer. \textit{Labeled} updates are for everything else.

In \RH, all communication is done through emails and authors do not explicitly tag them as updates. For this reason, we define an update on \RH as \emph{messages sent to a thread where the author has already participated once}.

Regarding update frequency in our sample, the \SO R tag contained 2.5 more updates than the \RH mailing list. Corrections are more common in \SO (almost 50\%), while \RH updates are often related to the adding of information to a thread (providing background, expansion, and explanation).

\paragraph*{Flags}
Flags are used to alert members of the community.

\SO contains a flagging mechanism, often used to get a moderator's attention. These flags can accomplish various objectives: mark a message as spam, rude, or abusive behavior; or identify
duplicate questions, off-topic messages, unclear questions, opinion-based questions, and low-quality
answers. Depending on the type of flag, this can lead to a thread being closed or the loss of user reputation points.  

\RH doesn't have a built-in flagging mechanism, however, \RH users utilize the concept of flags, which we define as \emph{messages used to call the attention of other community members}, similar to how flags are used in \SO.

% The main objective of Flags in both channels is to 
% keep a
% healthy community, promote discussion, and to raise specific
% issues.  %Flags we found in emails that also contained other types of information (such as answers, comments, and updates).

% Hance, flags in the \RH mailing list do not constrain users to answer questions, or to clarify
% what it is already asked.
% Under our definition, flags might be used by the person who asked or answered a question
% (in \SO authors of a question cannot add flags to it). 

In terms of their frequency, \SO flags are primarily used to mark repeated questions; in contrast, flags on \RH are used to indicate that a previous answer is incorrect. R tag posts on \SO had 1.5 times more flags than posts on the \RH mailing list.

\paragraph*{Comments}

On \SO, comments are considered ``temporary `Post-It' notes left on a question or answer''\footnote{\url{http://stackoverflow.com/help/privileges/comment}}. Comments are located below each question or answer and can be used as a follow-up to a question, or to answer or clarify a question. In \RH, we define comments as messages written to \emph{improve an answer or as a follow-up of a discussion}; Note, in order for an email to qualify as a Comment it should not be written by the person who asked or answered the original question (otherwise, the message would be considered an Update instead).
Because both \SO and \RH permit participants to ask multiple questions in the same thread, the sub-categories of Comments are not mutually exclusive.  

Regarding their frequency, the main difference between both channels is that in \SO r-tag Comments are less likely to be Corrections/Alternative than in \RH; also, the \SO r-tag sample had 2.1 times more Comments than the \RH sample.

\subsection{How Is Knowledge Constructed on \SO and the \RH Mailing List}
%\subsection{RQ-2. How is knowledge constructed on \SO and the \RH mailing list?}
\label{sec:rq2}

Through our analysis, we identified two different approaches for constructing knowledge (RQ2):

\begin{description}[itemsep=2pt, topsep=0pt, leftmargin=1em, parsep=0pt]
\item[Participatory knowledge construction] is an approach where answers are created through the cooperation of multiple users in the same thread. Participants complement each other's questions by discussing pros and
  cons of each answer, and by adding different viewpoints, additional information and examples.
  This process is similar to a team working together towards
  a common objective.

\item[Crowd knowledge construction] leverages the experiences of many users who work in a relatively
  independent manner. Each user contributes to the thread, adding variety to the pool of solutions. However, the priority of the users is to provide a correct answer, and not to discuss other solutions.
  This is comparable with the concept of a group in which people work towards the same objective but not necessarily together (e.g., Amazon's mechanical turk). Participants can vote over others ideas, but the
  idea is not constructed through a discussion process.
\end{description}

In the \RH mailing list, \textit{participatory knowledge} construction takes place when:
\begin{enumerate*}[label=(\arabic*)]
  \item previous answers are included in the current answer with clear link between them; or
  \item when a reply contains a direct reference to other answers or authors.
\end{enumerate*}
Figure \ref{fig:ML-PK1} depicts two examples of the way participatory knowledge occurs on the \RH mailing list: direct citation of the author of a previous answer, and inferable links between answers.

    
    \begin{figure}[!htb]
        \centering
        \includegraphics[width=\columnwidth]{Figures/ML-PKimg2}
        \includegraphics[width=\columnwidth]{Figures/ML-PKimg11}
        \caption[Participatory knowledge on the \RH mailing list.]{Participatory knowledge on the \RH mailing list.}
        \label{fig:ML-PK1}
      \vspace{-3mm}
    \end{figure}

On \SO, \textit{participatory knowledge} construction takes place when:
    \begin{enumerate*}[label=(\arabic*)]
    \item it is possible to infer a link between answers, via either a direct or indirect reference; or
    \item comments complement the answer, or directly cite another author.
    \end{enumerate*}
Additionally, on \SO participatory knowledge construction happens in different places, perhaps as a consequence of its rich interface. We observe this type of knowledge construction when a user answers a question and directly cites or links to someone else's  answer in the thread, or when a user cites someone else's question or answer in a comment (a typical case is linking to a previously asked question). Figure \ref{fig:SO-PK1} depicts an example of participatory knowledge construction on \SO: the user has added a comment to help another user when the answer is not sufficient.  In the comment, the user references another author's answer.

    \begin{figure}[!htb]
        \centering
        \includegraphics[width=\columnwidth]{Figures/SO-PKimg5}
        \caption{Example of participatory knowledge on \SO. Users build on the comments and answers of other users.}
        \label{fig:SO-PK1}
        \vspace{-3mm}
    \end{figure}

\textit{Crowd knowledge} construction is observable when:
  \begin{enumerate*}[label=(\arabic*)]
    \item there is no direct or inferable reference between answers,
    \item answers are a variation of one of the answers on the thread.
  \end{enumerate*}
Figure \ref{fig:CKC_MLSO} depicts an example of how crowd knowledge construction is visible on \SO. As can be seen from the figure, two of the three answers provided the same solution. 

    \begin{figure} [!htb]
        \centering
        \includegraphics[width=\columnwidth]{Figures/SO-CSimg2}
        \caption{Example of how crowd knowledge construction occurs. The three authors provide similar answers, but they do it independently of each other.}
        \label{fig:CKC_MLSO}
\vspace{-3mm}
    \end{figure}

\subsection{Why Users Post to a Particular Channel}
%\subsection{RQ-3. Why do certain users post to both \SO and the \RH mailing list}
Analyzing the archived data revealed that some users (79 cases in our sample) posted the same question on both \SO and \RH mailing list. To gain deeper insights on why R members would act this way, we conducted a survey. Based on the responses from the survey, we identified that being active on both channels brings benefits to those asking and answering questions (RQ3):

\begin{description}[itemsep=2pt, topsep=0pt, leftmargin=1em, parsep=0pt]
\item[Find a better answer:] As expected, two channels are better than one. 
  One channel might provide a better answer than the other.
\item[Support follow-up questions:] We found that \RH is often used to have follow-up discussions on
  specific answers provided to \SO questions. \SO focus is on finding an answer to a question, and does not
  provide an environment to discuss the specifics of an answer (unless it is asked as another question).
In contrast, in \RH, the discussion can continue long after an answer has been found with follow-up questions (not only by the person who asked the original question).  
\item[Speeds up answers:] Members ask the same question on both channels in order to get an answer faster. However, this behavior is not encouraged by the community as it is deemed impolite\footnote{\href{https://goo.gl/p9vVaj}{https://goo.gl/p9vVaj}}.
% `... it's impolite to cross post across several lists (i.e. stackoverflow and \RH)''
\end{description}


In contrast, some survey participants stated their preference of one channel over the other. We summarize their responses below.

\subsubsection{Why Participants Post on \SO}

Survey participants preferred \SO for five main reasons: (a) being able to gain peer recognition (the advantage of gaining points---and visibility---is a major draw towards SO); (b) its rich and user friendly interface; (c) answers are straight to the point; and, (d) questions are usually answered faster in \SO than \RH; and e) it is easy to search for previous questions and answers.

The main drawbacks of using \SO mentioned by the survey participants were: (a) an abundance of related questions; (b) certain level of experience is required to understand some of the answers; and, (c) \SO strict rules that only allow questions and their answers (\SO does not allow discussions nor questions about opinions).


\subsubsection{Why Participants Post on \RH}
\label{sec:rh}

The benefits of using \RH include: (a) the convenience of just handling email; (b)  following the mailing list provides awareness and increases learning in new topics; (c) the flexibility of \RH regarding the topics that one can discuss; and, d) the participation of highly experienced users. The main disadvantages of \RH are: (a) sometimes aggressive behavior arises during discussions; and, (b) searching the archives is not easy.


%%% Local Variables:
%%% mode: latex
%%% TeX-master: "knowledge-curation.tex"
%%% End:

% vim: set fenc=utf-8 ft=latex encoding=utf-8
% -*- mode: latex; coding: UTF-8; -*-
%!TEX root = knowledge-curation.tex
\section{Theory and Discussion}
\label{cha:theory}

    In this section we present a theory that encapsulates the observations and insights found during the analysis of the data, including, the comparison of the way knowledge is shared on both channels, and the recommendations of the use of Q\&A media channels.
    We also discuss them with respect to the related literature.

%Categorization of the way resources are used on Q\&A media channels.
\subsection{How the knowledge is shared on channels}

    Based on the categorization performed, both channels provide roughly the same knowledge support for questions and answers.
    However, there are some differences between both channels which are summarized in Table~\ref{table:constrat}.
    These observations are tendencies, and they are not behaviours unique of each channel.

    \begin{table}[!htb]
      \centering
      \caption{Comparison of the way knowledge is shared on Stack Overflow and the R-help mailing list.}
      \label{table:constrat}
      \begin{small}
          \setlength{\tabcolsep}{5pt}
          \begin{tabular}{@{}lll@{}}
            \toprule
            \textbf{}      & \textbf{Stack Overflow} & \textbf{R-help}\\
            \midrule
            Knowledge construction & Crowd             & Participatory \\
            Topic restriction      & Topic restriction & No topic restriction \\
            Knowledge              & Curated knowledge & Knowledge development\\
            \bottomrule
          \end{tabular}
      \end{small}
    \end{table}

    Stack Overflow is more suitable for questions with a clear answer, and the R-help mailing list is more suitable to discuss topics that are in and out of the software development domain.
    The study by Squire~\cite{Squire2015a} presents a similar difference between Stack Overflow and a mailing list of multiple projects through a different approach.
    
    The main benefit of using crowd knowledge construction is the existence of a pool of solutions, which combined with curation mechanisms, produces multiple ways to solve the same problem (diversity of solutions) in a clear solution that can be reused.

\subsubsection{Knowledge construction}

    Stack Overflow's gamification system encourages participation by giving points to those who participate~\cite{Singer2013}.
    Even when the diversity of answers provided in Stack Overflow is high, users tend to not contribute (edit or comment) in such answers; instead, some users provide their own answers.
    For instance, in the Stack Overflow thread \textit{\href{http://goo.gl/Mb4Pbk}{``Resources for learning SAS if you already familiar with R''}}, three of the six answerers referenced the same books.
    The gamification mechanism gives reputation to those who answer the questions, even when each extra answer might not add any new insight about how to solve a specific problem.
    Stack Overflow curation mechanism provides information about the popularity of answers, but not why, or how it is better than others answers.

    In contrast, the R-help mailing list tends to be more collaborative on how users construct knowledge, and discuss proposed answers.
    Participants of the R-help mailing list tend to provide more background to the answers as well as to explain answers of other participants.
    For example, the question \textit{``Arrange elements on a matrix according to rowSums + short `apply' Q''} was posted on both \href{http://goo.gl/a8AES8}{Stack Overflow} and \href{http://goo.gl/PGflT5}{R-help} mailing list.
    Both communities answered the question using a different knowledge construction approach.
    On Stack Overflow, each participant provided their own solution without any evidence of collaboration between them.
    Whereas users on the R-help mailing list complemented each other answers by providing extra information.

    The Stack Overflow's knowledge construction is not limited to crowd knowledge construction, it also presents collaborative ways to construct knowledge.
    However, the crowd one seems more prevalent.
    On the R-help mailing list happens the same as on Stack Overflow, but the other way around.

    Tausczik \textit{et al}.~\cite{Tausczik2014} found the same collaborative knowledge construction over another knowledge domain and channel, the mathematics domain on Math Overflow (a Q\&A channel focused on solving mathematical problems).
    Thus, we extend what Tausczik \textit{et al}. found to other media channels and domains.

    The findings presented here as theory can be used to identify how channels' features or community members might affect the construction of knowledge.
    For instance, we identified that gamification might affect collaboration between users. 
    Users prefer to create their own answer instead of collaborating with others.
    Additionally, it might be possible for indirect collaboration, like the one happening on the comments on Stack Overflow to improve discussion and participatory knowledge construction if there was a mechanism to provide points for this type of participation.
    However, more studies are required to extend our observation to other domains, communities, and channels.

\subsubsection{Topic restriction}

    One of the best ways in which the R-help mailing list complements Stack Overflow is on the topics that can be posted on both channels, and the format of the questions that can be answered. 
    On the R-help mailing list, questions related to R, but not focused on software development, are not rejected by the community (see section \ref{cha:findings-types} Flags).
    Also, topics that trigger a discussion, even when they are not related to software development, are welcomed in to the R-help mailing list.
    For instance, when users discuss the creation or improvement of the R community channels (see section \ref{sec:userbeh}); or when a question about installing R on \textit{Linux} is asked on the R-help mailing list (like {\href{http://goo.gl/1JLOUF}{\textit{``R on X11 under Linux''}}}).
    In contrast, on Stack Overflow, questions that trigger discussion are flagged as opinion-based, or as off-topic, and they might be closed. 
    For example, the questions \textit{\href{http://goo.gl/9JjZW1}{``What's a good example of really clean and clear [R] code, for pedagogical purposes?''}} was closed as off-topic because the question was not related to software development.

    As explained by, one of the participants discussing on \textit{\href{http://goo.gl/mTccwx}{``creating an equivalent of r-help on r.stackexchange.com?''}} commented:
    \begin{quote}
        \textit{``got an R programming question that you think has a definite answer? Post to [Stack Overflow]. Want to ask something for discussion, like what options there are for doing XYZ in R, or why lm() is faster than glm(), or why are these two numbers not equal-- post to R-help. Questions like that do get posted to [Stack Overflow], but we [moderated] them down for being off-topic and they disappear pretty quickly.''}
    \end{quote}

    We believe that is important to understand how the knowledge is constructed on media channels, and how different mechanisms such as gamification or topic restriction can affect the knowledge construction~\cite{Li2015}.
    Through this understanding, researchers can gain insights of how to support future media channels, and user diversity~\cite{Vasilescu2014b}.

\subsubsection{Curated knowledge and knowledge development}

    On the R-help mailing list questions tend to have more background than on Stack Overflow.
    The knowledge embedded in the R-help mailing list's answers can be used to learn new procedures, as well as identify the train of thought that guided participants when forming an answer.
    For instance, U26 explains:
    \begin{quote}
        \textit{``Many developers share my view that [Stack Overflow] is a very bad model, ... [it] removes the value added by reading list traffic that doesn't seem directly relevant to a currently conceptualised question, but which may lead to a new conceptualisation (out of the frame thinking). [Stack Overflow] cannot do that.''}
    \end{quote}
    Similarly, U35 explains that it uses the R-help mailing list if the questions are not 100\% \textit{``help-me-to-code-this''}.

    In contrast, Stack Overflow shines when questions have to be kept for posterity. 
    The curation mechanism provides tools to keep the channel clean of what seems to be unnecessary information (e.g., flagging questions, deleting comments, editing messages, and demoting irrelevant answers).

    \begin{quote}
        \textit{``[Stack Overflow] is an excellent model for providing a rich resource for users of R, which the R-Help mailing list was not. 
        Ability to include light markup, render code blocks nicely, [and] not [having] nested email threads all helps the experience of searching for and finding the help that a user needs, and I want to contribute to that.''} [U14]
    \end{quote}

    Thus, we identified that there are certain benefits for keeping the history of the question available.
    As U26 said, there are some benefits to reading what a user thinks is not important for conceptualized questions, but which may lead to out of the frame thinking. 

\subsection{Recommendations for using multiple Q\&A media channels}

    One of the interests of this study was to derive a set of recommendations for using media channels, as a mechanism to improve the benefits of their usage.
    Based on the analysis of the \textit{flags} (which are often used to point out users' behaviours); rules, manuals and FAQ resources from Stack Overflow and the R-help mailing list; threads that were posted in both channels (i.e., the same question by the same user in both channels); and the answers of the survey.
    From this data, we provide a set of 4 recommendations.

    By studying communities that migrated development support towards Stack Overflow, Squire~\cite{Squire2015a} found that the main reason for communities coming back to the mailing list are topic restriction and the question's format expected on Stack Overflow.
    Squire and us suggest that communities of practice should evaluate the real benefits of each channel before moving to newer technologies.

\subsubsection{Choose the correct channel}

    As described in section~\ref{cha:background}, media channels provide a list of \textit{topics} permitted, this are available either in the description of the channel or in their limitations.
    The control of topics is often regulated by the community or channel's moderators.
    U35 explains that \textit{``...Stack Overflow has more limited range of help topics (help for code only), whereas R-help is broader (philosophy, posting announcements, etc.)''}.
    Knowing what channel is \emph{more suitable for a specific topic can improve the response time or quality of the answer by taking advantage of the community members' knowledge}.

    Additionally, \emph{choosing the proper channel keeps the knowledge where it is most useful, thus enhancing the quality of the content of the channel}.
    For example, in the R-help's thread \textit{\href{http://goo.gl/EJHWrs}{``Bumps chart in R''}}, an user wrote: \textit{``(cross posting to the ggplot2 group for posterity) Here's how I'd approach it...''}, that is, cross-posting the question---previously posted and answered on the R-help list---in order to keep a record of the knowledge where it reaches more users, and where it is more useful to the community.

    In some cases, questions should be \emph{answered by a \textit{specific group} (e.g., r-core team) regardless of the topic}.
    U32 stated \textit{``If I really want an answer from someone in R-core or closely related people, I would definitely choose the mailing list''}.
    For example, in the R-help's thread \textit{\href{http://goo.gl/7olLv7}{``Cointegration and ECM in Package \{urca\}''}}, a participant asked the R-core team directly how to solve a problem: \textit{``Dear R Core Team, I am using package \{urca\} to do cointegration and estimate ECM model, but I have the following two problems...''}.

    In this scenario, \emph{websites of a specific package or library might be the best method to communicate directly with the creators of that technology}.
    In some cases, the description of the channel or package provides the necessary information, such as the maintainers or participants (e.g., R-help primary help webpage\footnote{\url{https://stat.ethz.ch/mailman/listinfo/r-help}}, \emph{rcpp} package\footnote{\url{https://cran.r-project.org/web/packages/Rcpp/index.html}}, or \textit{r-tag} info page on Stack Overflow\footnote{\url{http://stackoverflow.com/tags/r/info}}).
    Figure \ref{fig:CCchannel} depicts an example of how developers of a package can be reached using Stack Overflow (on the left) or by email (on the right). 

    \begin{figure} [!htb]
        \centering
        \includegraphics[width=\columnwidth]{Figures/CCchannel}
        \caption{Example of how to reach developers of the \emph{rcpp} package. On the left, Stack Overflow, and on the right, the \emph{rcpp} webiste.}
        \label{fig:CCchannel}
    \end{figure}

    Some channels are more suitable for certain \emph{type or format of questions}. 
    Thus, R-help mailing list is a place for discussion, and Stack Overflow is a place for questions that have a clear answer.


\subsubsection{Read the user manuals, channel rules and learn the basic concept of the technology used}

    Through the study, we noticed that most of the harsh responses from the community were give to users who did not read the posting guide or learned the basic concept for each technology (e.g., \textit{``An Introduction to R''}\footnote{\url{https://cran.r-project.org/doc/manuals/r-release/R-intro.pdf}})---users should demonstrate a minimum understanding and use of the programming language.
    The community expects that if someone wants to use a channel, they should learn about it in advance, and learn the basics of the technology that they are using.
    For instance, in the R-help's thread \textit{\href{http://goo.gl/Dc8gXw}{``Quantile''}} it is remarked the points of a guide that the user asking the question did not follow: \textit{``...Please read the Posting Guide. It asks that you not crosspost. If you post a followup to rhelp, then the reading of the Posting guide will tell you that much more in the way of detail about your setup was requested...''}.

    Depending on the channel, the amount of guide lines and posting guides available might differ. 
    Stack Overflow provides user manuals\footnote{\url{http://stackoverflow.com/help}} for each of the main features of the channel such as badges, questions, answers, flags, comments, and reputation system.
    In contrast, the R-help mailing list only has the general instructions\footnote{\url{https://www.r-project.org/mail.html\#instructions}} and the posting guide user manual\footnote{\url{https://www.r-project.org/posting-guide.html}}, which make the R-help mailing list a more user friendly environment for new users in terms of what users have to read in advance.

    Moreover, depending on the technology, there are some \textit{community} user manuals that might be useful to read before participating in the channel.
    For instance, the post on Stack Overflow \textit{``How to make a great R reproducible example?''}\footnote{\url{http://stackoverflow.com/questions/5963269/how-to-make-a-great-r-reproducible-example}} provides tips and tricks for creating a reproducible example using the R language.
    Another example is the channel related user manual written by Hadley Wickham.
    It provides some tips for posting on the R-help mailing lists: \textit{``...Before putting all of your code in an email, consider putting it on \href{http://gist.github.com/}{[GitHub Gist app]}. It will give your code nice syntax highlighting, and you don't have to worry about anything getting mangled by the email system...''}

    Finally, there are technology manuals like \textit{``An Introduction to R''}), and the FAQ webpages that are available to the public---most of the time free of charge, from which any user can learn the basic of each technology.
    For instance, the R community provides a compendium of PDF documents for new users on different languages\footnote{The R Manuals are available at \url{https://cran.r-project.org/}}.
    While on Stack Overflow, supported technologies are provisioned with webpages and links to free and paid materials\footnote{Materials available at \url{http://stackoverflow.com/tags/r/info}}.
    Members are able to reference these materials when needed, e.g., \textit{``...You may want to acquaint yourself with the 'An Introduction to R' manual that came with your R installation to learn more about indexing.''}

\subsubsection{Choose a channel according to the user experience}
            
    The variety of media channels can be overwhelming to decide in which channel post.
    It is advisable to learn the characteristics of the channels before making a decision.
    For example, using Stack Overflow has benefits such as low response time~\cite{Mamykina2011} and peer recognition~\cite{Singer2013}, but user manuals should be read prior participation.
    A bad reputation in the channel might affect users in real life~\cite{Singer2013}.
    U14 pointed that one of the biggest challenges of using Stack Overflow is learning the \emph{ethos} of the channel.
    
    Communities like R, have multiple channels with overlapping functionality.
    The R-help mailing list can be used for the same purpose as Stack Overflow, but it has a different audience, which might bring some benefits.
    For instance, the R-help mailing list is less confronting, it can be used to learn rather than just get the answer, and it can be sometimes friendlier.
    
    Squire's study~\cite{Squire2015a} as well as our findings suggest that Stack Overflow might not be enough to fully support software developers.
    As it is suggested by the amount of active users on the R-help mailing list, a community might require a place to discuss topics that are in and out of the software development domain.
    The fragmentation of topics within the Stack Exchange's Q\&A channels (each channel from Stack Exchange supports a small group of topics), the complex rules of their sites, and the gamification mechanism might be a difficult issue to handle for some users~\cite{Vasilescu2013}.

\subsubsection{Provide a background to the question}

    In spite of reading the documentation available, a user may fail to address the channel appropriately.
    The community may feel that the question asked, the information provided or something else is not in compliance with the expectations and rules of the channel.
    In such cases, one should describe the documentation read, the attempts made, and what is being trying to achieve.
    This would avoid answers like \textit{``read the manual''} or \textit{``read the posting guide''}, as well as helping the participants to help.
    As an example, in the thread \textit{\href{https://goo.gl/Gbek3R}{``lme4 GLMM''}}, a user asked \textit{``I'm very sorry for my repeated question, which I asked 2 weeks ago, namely: I'm interested in possibly simple random-part specification in the call...''}.

\subsubsection{User Behaviour}
\label{sec:userbeh}

While analysing questions and answers, we identified user behaviours that are not reflected in the categories we developed, but that we believe are worth
mentioning.  These behaviours provide evidence of their altruistic way of thinking and the strong commitment that users have within the community.

    \begin{packed_enum}
        \item \textbf{I answered my own question}: Some questions are answered by the same user who asked the question. They posted back to the channel to document their solution.
        For example, \textit{\href{http://goo.gl/FG59Mw}``I've discovered the answer to my own question.''}} or \textit{\href{https://goo.gl/r3z0DX}{``Just for the records (and if anyone ever wants to find the ``solution"), I solved my own problem.''}}

        \item \textbf{I did it for you}: When answering authors provide an extensive amount of source code to help others. For instance, \textit{\href{http://goo.gl/GXWGG3}{``I have coded up the algorithm from the Cameron and Turner paper. Dunno if it gives exactly the same results as my (Splus?) code from lo these many years ago...''}}.

        \item \textbf{Answered, updated or continued years later}: Some answers are provided months or years after the question was asked.
        For instance, \href{http://goo.gl/k6ZARR}{a user on Stack Overflow modified an answer to provide a more updated version of the source code}; and a \href{http://goo.gl/kgSHZv}{question asked on the R-help mailing list in 2012 was continued two years later}.

        \item \textbf{Ideas for improvement or creation of the channel}: This behaviour is specific for the R-help mailing list. Sometimes users suggest modifications or new features to improve the channel. For instance, a \href{http://goo.gl/p0IunD}{user proposes to create a package repository that can be accessible through a public wiki, or version control interface}.
    \end{packed_enum}

    We also identified 2 behaviours that might result in a bad response from the community:
    \begin{packed_enum}
        \item \textbf{Cross-posting:} The user posts the same question in both channels at the same time.
        For instance: \textit{\href{http://goo.gl/ENKrVK}{``-1 for cross posting to r-help – [user name]''}}.
        \item \textbf{Posting guidelines violation:} The user behaves in a way that it becomes apparent that they did not read the posting guidelines.
        For instance, a user asked a question that seems to be the opposite of what the posting guide recommend, and someone answered: \textit{\href{http://goo.gl/FUm1HC}{``...If you read the Posting Guide I think you will find precisely the opposite expectation explicitly presented. Using my "cheeky code" would only be part of the recommended actions to take before posting if you follow the recommendations of the "Do your homework before posting:"...''}}.
    \end{packed_enum}


\subsection{Recommendations for using external resources}

    A common practice to answer or ask questions is to provide links for documentation, examples, source code, or other resources.
    As links point to online resources that might not exist in the future, it is important to include the key points of the resource within the question or answer.
    For instance, when a question or answer contains information in an external file hosting service like Dropbox or Google Drive, the owner of the service account can break the link at any moment, leaving the message incomplete or impossible to reproduce, like the thread \textit{\href{http://goo.gl/5nanFU}{``Is it possible to create a 3d contour plot without continuous data in R?''}}.
    U33 suggested: \textit{``Questions should be self-contained as much as possible. Exceptions: recognizable links such as CRAN, R documentation, etc.''}.

    Based on our observations, we constructed a set of recommendations for links that are the exception of the rule:

    \begin{description}[itemsep=3pt, topsep=2pt, leftmargin=3em, parsep=0pt]
        \item[Well known websites] are expected to be maintained in the long term like Wikipedia, the official documentation in CRAN.
        For example, the Stack Overflow's thread \textit{``calculating convolution of multinomial distribution''} a user posted \textit{`I'm doing a simulation where I need to calculate a \href{https://en.wikipedia.org/wiki/Convolution_of_probability_distributions}{[Wikipedia convolution]} of \href{https://en.wikipedia.org/wiki/Multinomial_distribution}{[Wikipedia multinomial distributions]}...''}.

        \item[Resources that support or expand the message] when the important information is already in the message.
        For instance, the Stack Overflow's thread \textit{``How do I save all the draws from a MCMC posterior distribution to a file in R''} clarifies \textit{``...You should be able to open a text connection using ?file \href{http://stat.ethz.ch/R-manual/R-devel/library/base/html/connections.html}{[more information]} with the open argument set to write...''}.

        \item[Material relevant to the message is too big] as papers or demonstrations.
        For instance, the R-help's thread \textit{``Using FUNCTION to create usable objects''} a user stated \textit{``I suspect you are trying to find your way into Circle 6 of 'The R Inferno' but haven't yet got in. \href{http://www.burns-stat.com/pages/Tutor/R\_inferno.pdf}{[R Inferno]}''}.
    \end{description}

\section{Threats to validity}+
\label{cha:threats}

    In this section, we explain how we addressed each threat to
    validity.\remarks{Consider make the data available online.}

\subsection{Construct validity}

    The open coding of a large collection of messages may introduce bias.
    To minimize the bias by misinterpretation, we used multiple data sources to triangulate our findings (survey, documentation, and messages from two media channels), we randomly selected data, and two researchers performed the data coding.
    We applied the Cohen Kappa coefficient on categories that were not mutually exclusive, whose purpose  was to trigger discussion between coders.
    We set a threshold of 80\% as the minimum to obtain agreeable results, which is higher than the 60\% suggested in the literature~\cite{Landis1977}.
    We identified and reported the discrepancies and contradictions.

\subsection{Internal validity}

    To compare two different data sources, we mapped the message types between Stack Overflow and the R-help mailing.
    However, the R-help mailing list contains unstructured data, and some data may be unsuitable to compare as both data sets, R-help mailing list and Stack Overflow, have different time frames.
    To minimize unsuitable messages, we limited the analysis to the time frame that all the information was available in both data sets.
    Additionally, our understanding of that data and the observations we made played a big role in the mapping exercise, and it is subject to research bias.
    To minimized the bias, two researches performed the mapping.

\subsection{External validity}

    This study is an exploratory case study and only applies to the R community.
    A case study cannot be assumed to be generalizable until further evaluations have been conducted~\cite{Yin2009}.
    Findings in this research should be tested in other communities and with other channels to see if they apply to these other contexts.

    The R community is composed by atypical programmers.
    The R programming language is used to solve statistical problems, the final product is a script that process specific data.
    Most R language users are likely to be \textit{casual developers} with limited or non-existent programming experience, with backgrounds that vary from biologists to statisticians.
    As a consequence, this study might not represent the knowledge that a software developer community shares.

%%% Local Variables:
%%% mode: latex
%%% TeX-master: "knowledge-curation.tex"
%%% End:

% vim: set fenc=utf-8 ft=latex encoding=utf-8
% -*- mode: latex; coding: UTF-8; -*-
%!TEX root = knowledge-curation.tex
\section{Conclusions}
\label{cha:conclusion}

    Understanding the interplay between channels should be the next step to gain further insights into software development practices.
    To that end, we contrasted the way knowledge is shared on Stack Overflow and the R-help mailing list, as well as an extensible categorization of Q\&A channel messages that is meant to be used to compare and analyse knowledge in media channels.
    To conduct a fair comparison, we collected raw data from both channels, transformed this data to a common format through an analysis of the messages to map questions, answers, and other elements.

    After the analysis of more than 500 threads, we identified five types of messages: question, answer, comment, update, and flag; and more than 35 categories along with their properties, including how-to question, tutorial answer, announcement update, not-an-answer flag, and clarifications comment.

    We identified two mechanisms for the construction of knowledge: \emph{participatory}, in which answers are created with the collaboration of various users; and \emph{crowd-based}, which is non-collaborative and where solutions are posted without any acknowledgement to previous answers.
    We also analysed links and how these contribute to the construction of knowledge, such as by referencing source code, projects, and other posts related to a particular question.

    We also identified user behaviours based on their comments.
    We noted certain usage characteristics such as solving their own question by providing the answer, and cross-posting in both channels.

    To bring further insights on our findings, we conducted a survey that collected 26 answers from R users.
    The result was a list of pros and cons of using both channels. We incorporated them in the set of recommendations for using multiple channels and resources.

%%% Local Variables:
%%% mode: latex
%%% TeX-master: "knowledge-curation.tex"
%%% End:


\bibliographystyle{abbrv}
\bibliography{references}

\end{document}

%%% Local Variables:
%%% mode: latex
%%% TeX-master: t
%%% End:
