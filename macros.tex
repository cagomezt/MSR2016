\usepackage[T1]{fontenc}
\usepackage[utf8]{inputenc}
\synctex=-1 
\usepackage[english]{babel}

\usepackage[hidelinks]{hyperref} % back-referencing + hyperlinks

\usepackage{xspace}
\usepackage{subscript}
\usepackage{multirow}
\usepackage[font=small]{caption}
\captionsetup{skip=4pt}

\usepackage{subcaption}
\usepackage{graphicx}
\usepackage{comment}
\usepackage{tabularx}

\usepackage[normalem]{ulem}
\usepackage{fixltx2e}
\usepackage{booktabs}
\usepackage{pslatex}
%\usepackage{mathptmx,helvet,courier}
\usepackage[inline]{enumitem}
\usepackage{flushend}

\newcommand{\mytitle}{How a Developer Community Creates and Curates Knowledge : \\A comparative Study of Stack Overflow and Mailing Lists}
\newcommand{\myauthor}{Carlos Arturo Gómez Teshima et al.}
\newcommand{\mysubject}{Interplay of multiple media channels}
\newcommand{\mykeywords}{Knowledge curation, Case study, Media channels}

% PDF metadata. Disable when using latexdiff and enable for final version.
%\usepackage[unicode=true,
%            bookmarks=false,breaklinks=true,pdfborder={0 0 0},
%            backref=none,colorlinks=false]{hyperref}
\hypersetup{pdftitle={\mytitle},
             pdfauthor={\myauthor},
             pdfsubject={\mysubject},
             pdfkeywords={\mykeywords}
}


\newenvironment{packed_enum}{
\begin{enumerate}[itemsep=3pt, topsep=2pt, leftmargin=2.4em, parsep=0pt]
}{\end{enumerate}}

% Comments
% commands for inserting comments
\usepackage{color}
\usepackage{ifthen}
\newboolean{showcomments}
\setboolean{showcomments}{true} % toggle to show or hide comments
\ifthenelse{\boolean{showcomments}}
  {
		%\usepackage{showkeys} %Show lables and refs
		\newcommand{\nbb}[2]{
		% \fbox{\bfseries\sffamily\scriptsize#1}
		\fcolorbox{black}{yellow}{\bfseries\sffamily\scriptsize#1}
		{\sf$\blacktriangleright$\textcolor{blue}{\textit{#2}}$\blacktriangleleft$}
		% \marginpar{\fbox{\bfseries\sffamily#1}}
		}
        \newcommand{\dmg}[1]{{\color{blue}\emph{Daniel says: #1}}\xspace}
        \newcommand{\gpoo}[1]{{\color{red}\emph{Germán says: #1}}\xspace}
        \newcommand{\todo}[1]{\textcolor{red}{{\sc #1}}}
        \newcommand{\internalnote}[1]{\marginpar{\scriptsize note: #1}}
        \newcommand{\version}{\emph{\scriptsize{$-$\today$-$}}}
		\newcommand{\remarks}[1]{\color{red}[#1]\color{black}}
		\newcommand{\modified}[1]{\color{blue}[#1]\color{black}}
		\newcommand{\raw}{$\rightarrow$}
		\newcommand{\ins}[1]{\textcolor{blue}{\uline{#1}}} % please insert
		\newcommand{\del}[1]{\textcolor{red}{\sout{#1}}} % please delete
		\newcommand{\chg}[2]{\textcolor{red}{\sout{#1}}{\raw}\textcolor{blue}{\uline{#2}}} % please change
		\newcommand{\ugh}[1]{\textcolor{red}{\uwave{#1}}} % please rephrase
		\newcommand{\cn}{\textcolor{blue}{[citation needed]}\xspace}
  }
  {
        \newcommand{\dmg}[1]{}
        \newcommand{\gpoo}[1]{}
		\newcommand{\nbb}[2]{}
        \newcommand{\todo}[1]{}
        \newcommand{\internalnote}[1]{}
		\newcommand{\remarks}[1]{}
		\newcommand{\modified}[1]{#1}
		\newcommand{\version}{}
		\newcommand{\ugh}[1]{#1} % please rephrase
		\newcommand{\ins}[1]{#1} % please insert
		\newcommand{\del}[1]{} % please delete
		\newcommand{\chg}[2]{#2} % please change
		\newcommand{\cn}{}{}
  }

\newcommand{\channel}{communication channel\xspace}
\newcommand{\channels}{communication channels\xspace}
\newcommand{\Channel}{Communication channel\xspace}
\newcommand{\Channels}{Communication channels\xspace}

\newcommand{\SO}{Stack Overflow\xspace}
\newcommand{\ml}{mailing list\xspace}
\newcommand{\RH}{R-Help\xspace}

\newcommand{\rqa}{What types of knowledge are shared on Stack Overflow and the R-help mailing list within the R community?}
\newcommand{\rqb}{How is the knowledge constructed on Stack Overflow and the R-help mailing list?}
\newcommand{\rqc}{Why do certain users post to both Stack Overflow and the R-help mailing list?}
%\newcommand{\rqd}{What are the advantages and disadvantages of using Stack Overlow or the R-help mailing list?}

\newcommand{\reca}{Choose the correct channel}
\newcommand{\recb}{Be aware of the channel rules and the basic concepts and nomenclature used}
\newcommand{\recc}{Provide good background to the question}
\newcommand{\recd}{Learn to use external resources}
\newcommand{\rece}{Act altruistically}




%%% Local Variables:
%%% mode: latex
%%% TeX-master: "knowledge-curation.tex"
%%% End:
