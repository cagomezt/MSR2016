%!TEX root = knowledge-curation.tex
\section{that other section}

The R language continues to grow in popularity, and with it, the size of its community of practice. The recent TIOBE index for Programming Languages ranks it
number 19 among all languages.  According to a recent survey, R has become the highest paying skill in IT. The interest in R continues to grow.
Early in 2016, Microsoft announced support for R in Visual Studio~\cite{RMicrosoft2016}.

Being an open source project, without commercial backing, the R community has played an important role in its diffusion. Those new to the R language have
numerous resources to learn the language and receive help: mailing lists, blogs, books, online- and offine--courses, questions-and-answers sites (such as
\SO). In all, these resources provide a vast and rich corpus of knowledge. The R-community benefit from this corpus, but it also the one that drives its
creation and curation.  For example, in July of 2009, during the Open Source Convention (OSCON'09), the organizers of a birds-of-a-feather session invited users
to participate in a Flashmob to seed Stackoverflow with R related questions~\cite{OSCONRFlashMob2009}. The premise of the session was that \SO lacked R-related
content. The organizers have gathered a list of commonly asked questions from the R-help mailing list, Rseek.org and a survey of R-users. Its impact was
noticed, and \SO acknowledge to ``officially condone'' the practice~\cite{SOFlashMob2009}.

The size of the R community has a network effect. As the number of its members grows, the awareness of the language increases, jobs are created, available
resources increase their size, depth and quality, new tools and libraries appear. In essence, helping individuals cope with R benefits the entire R
community: its members learn and/or improve their expertise, the experts gain reputation for their knowledge and willingness to help, the R-users base grows and
with it, its cloud and market.

Without a single entity directing and controlling the R-language, knowlege in R has grown organically from its community. Knowledge is exchanged and curated in
many \channels (emails, blogs, books, presentations, web sites, etc). Like any other community of practice, the R-community takes advantage of available \channels
to achieve this goal.
Two \channels are at the center of this process: R-help mailing list and \SO. The R-users mailing list was established \dmg{when} as way to assist those using
the language. \SO is not specifically oriented towards R, but its section dedicated to the language has grown rapidly\footnote{\href{http://r-bloggers.com/r-is-the-fastest-growing-language-on-stackoverflow/}{http://r-bloggers.com/r-is-the-fastest-growing-language-on-stackoverflow/}}.

Without a doubt, \SO has changed, for the better, the way programmers seek knowledge. \SO can play the role of a expert-on-call, who is capable---and willing--
to answer questions of any difficulty level about any programming technology (R included). The gamification features of \SO have also guaranteed the willingness of experts to answer
those questions---frequently within minutes of being posted. Equally important is the ability of \SO users to curate the knowledge being created, making sure that
the best answers surface to the top, and become a valuable asset to those seeking the same answer in the future. \SO has become an effective tool to create, curate and exchange knowledge, including knowlege or R.

One would expect that the traffic in the R-users \ml would have begin to fizzle as \SO popularity increses. If \SO is so effective at matching those who seek
knowlege with those that have it, doesn't that obviate the need for the R-users mailing list? at least regarding questions and answers. Yet, that does not appear
to be the case. The R-help \ml continues to grow in traffic, implying that there it is still an important resource for the R-community. It appears as if R-users
and \SO complement each other.

There exist obvious inherent differences between both \channels. Mailing lists unite users by subscription, creating a tight community, and their content lacks
organization (except for its natural organization provided by the metadata of the emails---subject, threading, authors, date) and are not optimized for long
term storage and retrieval.  \SO, on the other hand, is a more loose community and it is opmitized for the curation and long term storage of the knowledge.

However, little is known of the actual differences of use between both \channels. 
In particular, how the types questions-and-answers seeked in one channel compare to
the other, why users choose one channel over the other, why some users participate in both channels and what are the perceptions that its participants have
regarding each \channel.

The objective of this study is to emprically compare how knowledge, specifically knowledge manifested as questions-and-answers, is seeked, shared and curated in
both, the R-users mailing list and the R section of \SO.

\dmg{needs more}



%%% Local Variables:
%%% mode: latex
%%% TeX-master: "knowledge-curation.tex"
%%% End:
