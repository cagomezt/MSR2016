% vim: set fenc=utf-8 ft=latex encoding=utf-8
% -*- mode: latex; coding: UTF-8; -*-
%!TEX root = knowledge-curation.tex
\section{Introduction}
\label{cha:introduction}
The adoption and emergence of socially enabled tools and channels (e.g., GitHub, \SO, \ml) foster the formation of very large \textit{communities of practice} where members share a common interest, such as programming languages, frameworks, and tools~\cite{Storey2014}. These communities of practice rely and use many different communication channels, however little is known about how these communities create, share, and curate knowledge using such channels. One prominent community of practice is the R community.

The R programming language, being an open source project and without commercial backing, relies heavily on its rapidly growing and highly heterogeneous software development community. The R community plays an important role in its diffusion; members have numerous resources to learn the language and receive help: mailing lists, blogs, books, online- and offine- courses, questions \& answers sites (e.g., \SO). The R-community benefits from this vast and rich corpus of knowledge, but more importantly it also the one that drives its creation and curation.

Without a single entity directing and controlling the R language, knowledge in R has grown organically from its community. Similarly to other communities of practice, knowledge is exchanged and curated in many \channels. Two \channels are at the center of this process: \textit{R-help mailing list} and \textit{\SO}. The R-help mailing list was created as way to assist those using the language, while \SO is not specifically oriented towards R, but its section dedicated to R has grown rapidly\footnote{\href{http://www.r-bloggers.com/r-is-the-fastest-growing-language-on-stackoverflow/}{http://www.r-bloggers.com/r-is-the-fastest-growing-language-on-stackoverflow/}}.

Without a doubt, \SO has changed, for the better, the way programmers seek knowledge~\cn. \SO assumes the role of an ``expert-on-call'', who is capable---and willing---to answer questions of any difficulty level about any programming technology (R included). The gamification features of \SO guarantee the willingness of experts to answer those questions---frequently within minutes of being posted~\cite{Mamykina2011}. Equally important is the ability of \SO users to curate the knowledge being created, making sure that the best answers surface to the top, and become a valuable asset to those seeking an answer in the future. \SO has become a popular and effective tool to create, curate, and exchange knowledge, including knowledge of R.

One would expect that the traffic in the R-users \ml would begin to fizzle as \SO popularity increases. If \SO is so effective at matching those who seek knowledge with those that have it, doesn't that obviate the need for the R-users mailing list? at least regarding questions and answers. Yet, that does not appear to be the case. The R-help \ml continues to grow in traffic, implying that it is still an important resource for the R-community. It appears as if R-users and \SO complement each other.

There exist obvious inherent differences between both \channels. Mailing lists unite users by subscription, creating a tight community, and their content lacks organization (except for its natural organization provided by the metadata of the emails---subject, threading, authors, date) and are not optimized for long term storage and retrieval. On the other hand, \SO is a more loose community and it is optimized for the curation and long term storage of the knowledge.

However, little is known of the actual differences of use between both \channels. In particular, how the types of questions-and-answers seeked in one channel compare to the other, why users choose one channel over the other, why some users participate in both channels, and what are the perceptions that its participants have regarding each \channel.



The objective of this study is to empirically compare how knowledge, specifically knowledge manifested as questions-and-answers, is seeked, shared, and curated in both, the R-users mailing list and the R section of \SO. We apply a qualitative \textit{exploratory case study} methodology to answer the following research questions:

\begin{enumerate}[label=\bfseries{RQ-\arabic*.},itemsep=3pt, topsep=2pt, leftmargin=3em, parsep=0pt]
        \item \rqa
        \item \rqb
        \item \rqc
%        \item \rqd
    \end{enumerate}

\dmg{Add a summary of the results... to be done}



%%% Local Variables:
%%% mode: latex
%%% TeX-master: "knowledge-curation.tex"
%%% End:
