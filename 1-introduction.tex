% vim: set fenc=utf-8 ft=latex encoding=utf-8
% -*- mode: latex; coding: UTF-8; -*-
%!TEX root = knowledge-curation.tex
\section{Introduction}
\label{cha:introduction}
The adoption and emergence of socially enabled tools and channels (e.g., GitHub, \SO, mailing lists) foster the formation of very large \textit{communities of practice} where members share a common interest, such as programming languages, frameworks, and tools~\cite{Storey2014}. These communities of practice rely on and use many different communication channels, however, little is known about how these communities create, share, and curate knowledge using such channels. One prominent community of practice is the R community.

The R programming language is an open source project without commercial backing that relies heavily on its rapidly growing and highly heterogeneous software development community. The R community plays an important role in the diffusion of the R language; members have numerous resources for learning the language and receiving help, such as mailing lists, blogs, books, online and offine courses, and question \& answer sites (e.g., \SO). While the R community benefits from this vast and rich corpus of knowledge, it also drives the creation and curation of the information.

Without a single entity directing and controlling it, the R language has grown organically from its community. Similar to other communities of practice, knowledge is exchanged and curated in many \channels, and two particular \channels are at the center of this process: the \textit{\RH mailing list} and \textit{\SO}. The \RH mailing list was created as a way to assist those using the language, while \SO is not specifically oriented towards R, but its section dedicated to R (the R tag) has grown rapidly\footnote{\href{http://www.r-bloggers.com/r-is-the-fastest-growing-language-on-stackoverflow/}{http://www.r-bloggers.com/r-is-the-fastest-growing-language-on-stackoverflow/}}.

\SO has revolutionized the way programmers seek knowledge~\cite{li2013help,Vasilescu2014c}, assuming the role of a capable ``expert on call'' that is able---and willing---to answer questions of any level of difficulty about any programming technology (R included). \SO's gamification features guarantee that enthusiastic experts will answer questions, often within minutes of being posted~\cite{Mamykina2011}. Equally important is the ability of \SO's users to curate the knowledge being created, making sure that the best answers surface to the top and become a valuable asset to those seeking an answer now or in the future. \SO has become a popular and effective tool for creating, curating, and exchanging knowledge, including knowledge of R.

One would expect that the traffic in the \RH mailing list would begin to fizzle as \SO popularity increased. If \SO is so effective at matching those who seek knowledge with those that have it, doesn't that obviate most of the need for the \RH mailing list? Yet, that does not appear to be the case as the \RH mailing list continues to grow in traffic, implying that it is still an important resource for the R-community. It even appears as if the mailing list and \SO complement each other.

There are obvious inherent differences between both \channels. Mailing lists unite users by subscription, creating a tight community. Their content lacks organization, except for the natural organization provided by email metadata (e.g.,subjects, threading, authors, dates), and they are not optimized for long-term storage and retrieval. On the other hand, \SO is a more loose community and it is optimized for the curation and long-term storage of the knowledge.\cassie{I wouldn't say SO is loose...their rules and almost bully-like members are quite rigid. Perhaps Carlos meant something else here?} However, little is known about the differences in how people use both \channels, such as how the types of questions and answers sought in one channel compare to the other, why users choose one channel over the other, why some users participate in both channels, and how their participants perceive each \channel.



In this paper, we empirically compare how knowledge, specifically knowledge manifested as questions and answers, is sought, shared, and curated in both the \RH mailing list and on \SO. We applied a qualitative \textit{exploratory case study} methodology to answer the following research questions:

\begin{enumerate}[label=\bfseries{RQ\arabic*.},itemsep=3pt, topsep=2pt, leftmargin=3em, parsep=0pt]
        \item \rqa
        \item \rqb
        \item \rqc
%        \item \rqd
    \end{enumerate}

We identified and categorized the main types of knowledge contained in the \RH mailing list and in \SO messages (RQ1). The emerging categories (see Table~\ref{table:type-of-knowledge}) formed a typology that allows to study and characterize Q\&A knowledge dissemination within a community of practice. We utilized the typology to study how knowledge is constructed and shared on \SO and in the \RH mailing list. We found that these channels support two distinct approaches for constructing knowledge, \textit{participatory knowledge construction} and \textit{crowd knowledge construction}, however each channel supports them differently (RQ2). Our findings indicate that participatory knowledge construction is more prevalent on \RH, while crowd knowledge construction is more prevalent on \SO. Interestingly, we also found that some developers are active on both channels, and in some cases, even post the same questions. As a result, we investigated the benefits they gain by doing so (RQ3). We conclude the paper by providing recommendations for using different communication channels, and discuss how channel affordances and community rules (e.g., topic restriction, gamification) influence knowledge construction and curation.



%%% Local Variables:
%%% mode: latex
%%% TeX-master: "knowledge-curation.tex"
%%% End:
