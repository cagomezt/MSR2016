% vim: set fenc=utf-8 ft=latex encoding=utf-8
% -*- mode: latex; coding: UTF-8; -*-
%!TEX root = knowledge-curation.tex
\section{Introduction}
\label{cha:introduction}


    % Role of media channels
    Media channels play an important role in today's knowledge economy, as well as the collaboration, coordination, and communication activities that occur between programmers.
    Media channels are more than just delivery systems---they connect users with a \textit{community of practice} or groups of people with a common interest.
    Two popular media channels used by software developer are Stack Overflow\footnote{\url{http://stackoverflow.com/}} and mailing lists.
    Selecting the most appropriate media channel to transmit an idea can be challenging, given the variety of equally suitable tools and sites.
    To decide, a programmer considers the characteristics will benefit most among channels.
    Such considerations are: experts on the channel, flexibility on topics allowed, if the channel is asynchronous, socially enabled, or has gamification elements~\cite{Vasilescu2014c}.

    We investigated the way \textit{knowledge} (or user generated content) is curated within a particular software development community.
    For this study we chose the R community, since it provided broader relevance outside the software development community by including users with no or limited programming experience (e.g., biologist or statisticians).
    Our overarching goal was to provide tools for further studies that analyse and compare the knowledge flowing through media channels.
    Thus, the research question investigated are:

\dmg{what about Rephrasing RQ to stress differences between channels also?}
    \begin{enumerate}[label=\bfseries{RQ-\arabic*.},itemsep=3pt, topsep=2pt, leftmargin=3em, parsep=0pt]
        \item What types of knowledge are shared on Stack Overflow and the R-help mailing list within the R community?
        \item How is the knowledge constructed on Stack Overflow and the R-help mailing list? 
        \item Why do certain users post to both Stack Overflow and the R-help mailing list?
    \end{enumerate}

    We analysed the main Q\&A channels related to programming that the R community contains: R-help mailing list and Stack Overlow.
    To analyse the channels, we applied a qualitative \textit{exploratory case study} methodology.
    We also conducted a survey to bring further insights on the findings.
    We constructed a series of categories that supports knowledge classification and knowledge comparison of the main type of messages which these two channels provided.
    Based on the knowledge categories analysis, we compared the way knowledge was shared on Stack Overflow and the R-help mailing list.
    Finally, we extracted a set of recommendations to assist in the usage of multiple Q\&A channels, and when linking resources that are external to both channels.

\dmg{Add a summary of the results}

%%% Local Variables:
%%% mode: latex
%%% TeX-master: "knowledge-curation.tex"
%%% End:
