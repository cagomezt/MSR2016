% vim: set fenc=utf-8 ft=latex encoding=utf-8
% -*- mode: latex; coding: UTF-8; -*-
%!TEX root = knowledge-curation.tex
\section{Introduction}
\label{cha:introduction}

    % Role of media channels
    Communication channels play an important role in the collaboration, coordination, and communication activities that occur between programmers.
    Communication channels are more than just delivery systems, they connect users with a \textit{community of practice} or groups of people with a common interest.
    Two popular communication channels used by software developer are Stack Overflow\footnote{\url{http://stackoverflow.com/}} and mailing lists.
    To decide, a programmer considers the characteristics will benefit most among channels.
\dmg{This should be changed from the challenge of the developer to the challenge of the community: which one to have?}
    Such considerations are: experts on the channel, flexibility on topics allowed, if the channel is asynchronous, socially enabled, or has gamification elements~\cite{Vasilescu2014c}.

    We investigated the way \textit{knowledge} (or user generated content) is curated within a particular software development community.
    For this study we chose the R community, mainly because it is a fast growing, highly heterogeneous software development community with many users with
    limited software programming experience.
\dmg{Revise this goal}
    The main research goal of this paper is to understand and compare how a community uses its two most important communication channels: its mailing list and
    stack overflow. \dmg{needs sentence to link to the next list of research questions, use next sentece}

    We analysed the main Q\&A channels related to programming that the R community contains: R-help mailing list and Stack Overlow.
    We also conducted a survey to bring further insights on the findings.
    We constructed a series of categories that supports knowledge classification and knowledge comparison of the main type of messages which these two channels provided.
    Based on the knowledge categories analysis, we compared the way knowledge was shared on Stack Overflow and the R-help mailing list.
    Finally, we extracted a set of recommendations to assist in the usage of multiple Q\&A channels, and when linking resources that are external to both channels.

    To analyse the channels, we applied a qualitative \textit{exploratory case study} methodology.


\begin{enumerate}[label=\bfseries{RQ-\arabic*.},itemsep=3pt, topsep=2pt, leftmargin=3em, parsep=0pt]
        \item \rqa
        \item \rqb
        \item \rqc
%        \item \rqd
    \end{enumerate}

\dmg{Add a summary of the results... to be done}



%%% Local Variables:
%%% mode: latex
%%% TeX-master: "knowledge-curation.tex"
%%% End:
